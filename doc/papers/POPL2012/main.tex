\documentclass[9pt,authoryear]{sigplanconf}

\usepackage{ntheorem}
\usepackage{amsmath}
%\usepackage{amsthm}
\usepackage{amssymb}
\newtheorem{theorem}{Theorem}
\newtheorem{definition}{Definition}
\usepackage{preuve-modeste}
\usepackage{whizzy}
\usepackage{mathpartir}
\usepackage{macros}

\begin{document}

\conferenceinfo{POPL '12}{January 25-27 2012, Philadelphia, USA.} 
\copyrightyear{2012} 
\copyrightdata{[to be supplied]} 

\titlebanner{banner above paper title}        % These are ignored unless
\preprintfooter{short description of paper}   % 'preprint' option specified.

\title{Incremental Type Checking, Safely}
%\subtitle{Subtitle Text, if any}

\authorinfo{Name1}
           {Affiliation1}
           {Email1}
\authorinfo{Name2\and Name3}
           {Affiliation2/3}
           {Email2/3}

\maketitle

\begin{abstract}
This is the text of the abstract.
\end{abstract}

\category{CR-number}{subcategory}{third-level}

\terms
term1, term2

\keywords
keyword1, keyword2

% Généralités sur le problème et liste de nos contributions.
\section{Introduction}

% Explication informelle du problème et de notre solution.
\section{Incremental type checking}

% Partie technique
\section{The {\system} representational language}

\subsection{Syntax and interpretation}

\subsubsection{Syntax}

\paragraph{Conventions} 
% FIXME: Explain \listof and give syntax to concatenation.

\begin{figure}

\begin{center}
\begin{math}
\begin{array}{rclr}
&&&\bnfcomment{\bf Objects}\\
\term 

&::=& 
\tapp{\head}{\spine} 
& \bnfcomment{Application}
\\

&|&
\tlam{\binding}{\term}
& \bnfcomment{Abstraction}
\\

&|&
\tdef{\listof{\adef}}{\term}
& \bnfcomment{Local definitions}
\\

&|&
\topen{\term}{\var}
& \bnfcomment{Open}
\\
\\
&&&\bnfcomment{\bf Types}\\
\type

&::=& 
\typrod{\binding}{\type}
& \bnfcomment{Dependent product}
\\

&|& 
\tyapp{\tyconstant}{\spine}
& \bnfcomment{Type construction}
\\

&|& 
\tydef{\adefs}{\type}
& \bnfcomment{Local definitions}
\\


&|& 
\tyopen{\type}{\var}
& \bnfcomment{Open}
\\
\\
&&&\bnfcomment{\bf Kinds}\\
\kind

&::=&
\ktype
& \bnfcomment{Type}
\\

&|&
\kprod{\binding}{\kind}
& \bnfcomment{Dependent kind}
\\
\\

&&&\bnfcomment{\bf Heads}\\
\head

&::=& 
\constant
& \bnfcomment{Constant}
\\
&|& 
\var
& \bnfcomment{Variable}
\\
\\
&&&\bnfcomment{\bf Spine}\\
\spine

&::=& \emptylist
& \bnfcomment{Empty}
\\
&|& \head\,\spine
& \bnfcomment{Constructed}
\\
\\
&&&\bnfcomment{\bf Strictness promises}\\
\promise 

&::=&
\prigid
& \bnfcomment{Used in rigid position only}
\\

&|&
\pany
& \bnfcomment{Used in any position}
\\
\\
\adef 
&::=& 
\xadef{\var}{\term}{\type} 
& \bnfcomment{\bf Definitions}
\\
\adefs
&::=&
\emptylist
& \bnfcomment{Empty}
\\
& | &
\adef\, \adefs
& \bnfcomment{Constructed}
\\
\\
\binding
&::=& \bind{\var}{\promise}{\type}
& \bnfcomment{\bf Bindings}
\\
\\
&&&\bnfcomment{\bf Environments}\\
\tyenv 

&::=&
\tenvempty
& \bnfcomment{Empty environment}
\\

&|&
\tenvcons{\tyenv}{\binding}
& \bnfcomment{Bind}
\\
&|&
\tenvcons{\tyenv}{\adef}
& \bnfcomment{Define}
\\

% FIXME: Signature?
\end{array}
\end{math}
\end{center}

\caption{The {\system} representational language (Syntax).}
\label{fig:syntax-s2}
\end{figure}


% FIXME: Explain each construction of the syntax.

\subsubsection{Interpretation by expansion}

\paragraph{Spine calculus}
\begin{figure}
\begin{center}
\begin{math}
\begin{array}{rclr}
&&&\bnfcomment{\bf Objects}\\
\slfterm

&::=& 
\tapp{\head}{\slfspine} 
& \bnfcomment{Head application}
\\
&|& 
\tapp{\slfterm}{\slfspine} 
& \bnfcomment{Application}
\\

&|&
\tlam{\slfbind{\var}{\slftype}}{\slfterm}
& \bnfcomment{Abstraction}
\\
\\
&&&\bnfcomment{\bf Spines}\\
\slfspine

&::=&
\emptylist
& \bnfcomment{Empty}
\\
&::=&
\conslist{\slfterm}{\slfspine}
& \bnfcomment{Constructed}
\\
\\
&&&\bnfcomment{\bf Types}\\
\slftype

&::=&
\typrod{\slfbind{\var}{\slftype}}{\slftype}
& \bnfcomment{Dependent product}
\\

&|&
\tyapp{\tyconstant}{\slfspine}
& \bnfcomment{Type construction}
\\
\\
&&&\bnfcomment{\bf Kinds}\\
\slfkind

&::=&
\ktype
& \bnfcomment{Type}
\\

&|&
\kprod{\binding}{\slfkind}
& \bnfcomment{Dependent kind}
\\
\\

% FIXME: Signature?
\end{array}
\end{math}
\end{center}
\caption{Spine calculus (Syntax).}
\label{fig:syntax-s2}
\end{figure}


\paragraph{Importing shared definitions into local scope}
\begin{figure}
\boxed{
\openscope{\tyenv}{\term}{\listof\adef}
}
\begin{center}
\begin{mathpar}
\infer{
\openscopespine{\tyenv}
               {\type}
               {\term}
               {\emptylist}
               {\listof\adef}
}{
\openscope{\tyenv}{\oftype{\term}{\type}}{\listof\adef}
}

% FIXME: Les \' ne se voient pas tres bien.
\end{mathpar}
\end{center}
\boxed{
\openscopespine{\tyenv}{\type}{\term}{\spine}{\listof\adef}
}
\begin{center}
\begin{mathpar}
\infer{
\openscopespine{\tenvdef{\tyenv}{\var}{\head}{\type}}{\typebis}
               {\term}
               {\spine}
               {\listof\adef}
}{
\openscopespine{\tyenv}{\typrod{\bind{\var}{\promise}{\type}}{\typebis}}
               {\tlam{\bind{\var}{\promise}{\type}}{\term}}
               {\head\spine}
               {\xadef{\var}{\head}{\type}\listof\adef}
}

\infer{
}{
\openscopespine{\tyenv}
               {\type}
               {\tlam{\binding}{\term}}
               {\emptylist}
               {\emptylist}
}

\infer{
\openscope{\tyenv}{\tenvlookup{\tyenv}{\var}}{\listof\adef} \\
\openscopespine{\tenvcons{\tyenv}{\listof\adef}}{\type}
               {\term}
               {\spine}
               {{\listof\adef}'}
}{
\openscopespine{\tyenv}{\type}
               {\topen{\term}{\var}}
               {\spine}
               {\listof\adef{\listof\adef}'}
}

\infer{
\openscopespine{\tenvcons{\tyenv}{\listof\adef}}
               {\type}
               {\term}
               {\spine}
               {{\listof{\adef}}'}
}{
\openscopespine{\tyenv}
               {\type}
               {\tdef{\listof\adef}{\term}}
               {\spine}
               {\listof\adef {\listof{\adef}}'}
}

\infer{
}{
\openscopespine{\tyenv}{\type}
               {\tapp{\constant}{\spine}}
               {{\spine}'}
               {\emptylist}
}

\infer{
\tenvlookup{\tyenv}{\var} = \bind{\var}{\promise}{\type}
}{
\openscopespine{\tyenv}{\type}
               {\tapp{\var}{\spine}}
               {{\spine}'}
               {\emptylist}
}

\infer{
\tenvlookup{\tyenv}{\var} = (\oftype{\term}{\type}) \\\\
\openscopespine{\tyenv}
               {\type}
               {\term}
               {\spine{\spine}'}
               {\listof\adef}
}{
  \openscopespine{\tyenv}
                 {\type}
                 {\tapp{\var}{\spine}}
                 {{\spine}'}
                 {\listof\adef}
}


\end{mathpar}
\end{center}


\caption{Importation of a set of definitions from an object.}
\label{fig:expansion}
\end{figure}


\paragraph{Expansion of {\system} objects into spine calculus objects}
\begin{figure}
\begin{center}
\begin{math}
\begin{array}{rclr}

\interpret{\tapp{\head}{\spine}}{\tyenv}
&=& 
\tapp
    {\interpret{\head}{\tyenv}}
    {\interpret{\spine}{\tyenv}} 
\\
\interpret
    {\tlam{\bind{\var}{\promise}{\type}}{\term}}
    {\tyenv}
&=&
\tlam{\slfbind{\var}{\interpret{\type}{\tyenv}}}
     {\interpret{\term}{\tenvbind{\tyenv}{\var}{\promise}{\type}}}
\\
\interpret
    {\tdef{\adefs}{\term}}
    {\tyenv}
&=&
\interpret{\term}{\tenvcons\tyenv{\adefs}}
\\
\interpret{\topen{\term}{\var}}{\tyenv}
&=&
\interpret{\term}{\tenvcons\tyenv{\adefs}} \\
&&\defcomment{where } \openscope{\tyenv}{\tenvlookup{\tyenv}{\var}}{\adefs}
\\
\interpret{\constant}{\tyenv} 
&=&
\constant
\\
\interpret{\var}{\tyenv} 
&=&
\interpret{\term}{\tyenv} \\
&& \defcomment{where } \tenvlookup{\tyenv}{x} = (\oftype{\term}{\type})
\\
\interpret{\typrod{\bind{\var}{\promise}{\type}}{\typebis}}{\tyenv}
&=&
\typrod{\slfbind{\var}{\interpret{\type}{\tyenv}}}
     {\interpret{\typebis}{\tenvbind{\tyenv}{\var}{\promise}{\type}}}
\\
\interpret{\tyapp{\tyconstant}{\spine}}
          {\tyenv}
&=&
\tyapp
    {\tyconstant}
    {\interpret{\spine}{\tyenv}} 
\\
\interpret
    {\tydef{\adefs}{\type}}
    {\tyenv}
&=&
\interpret{\type}{\tenvcons\tyenv{\adefs}}
\\
\interpret{\tyopen{\type}{\var}}{\tyenv}
&=&
\interpret{\type}{\tenvcons\tyenv{\adefs}}\\
&&\defcomment{where } \openscope{\tyenv}{\tenvlookup{\tyenv}{\var}}{\adefs}
\\
% FIXME: Manque le cas des \listof.
\end{array}
\end{math}
\end{center}
\caption{Expansion.}
\label{fig:expansion}
\end{figure}


% FIXME: tout ceci n'a de sens que si openscope termine. 
% FIXME: ceci est garanti, en particulier, par le systeme de type qui suit. 

\subsection{Well-shared pretyped terms}

\subsubsection{A simply typed system for well-shared objects}

\begin{figure}

\begin{center}
\begin{math}
\begin{array}{rclr}
&&&\bnfcomment{\bf Pretypes}\\
\pretype

&::=& 
\ptconst
&\bnfcomment{Constant} 
\\

&|& 
\ptarrow{\pretype}{\promise}{\pretype}
&\bnfcomment{Arrow}
\\
\\
&&&\bnfcomment{\bf Prekinds}\\
\prekind

&::=& 
\pkconst
&\bnfcomment{Type} 
\\

&|& 
\pkarrow{\pretype}{\promise}{\prekind}
&\bnfcomment{Arrow}
\\
\end{array}
\end{math}
\end{center}
\caption{Syntax for pretypes and prekinds.}
\end{figure}


\begin{figure}

\begin{center}
\begin{math}
\begin{array}{rclr}
\erase{\tyapp{\tyconstant}{\spine}}
&=&
\ptconst
\\

\erase{\typrod{\bind{\var}{\promise}{\type}}{\typebis}}
&=&
\ptarrow{\erase{\type}}{\promise}{\erase{\typebis}}
\\

\erase{\tyopen{\type}{\var}}
&=&
\erase{\type}
\\

\erase{\tydef{\adefs}{\type}}
&=&
\erase{\type}
\\

\erase{\ktype}
&=&
\ktype
\\

\erase{\kprod{\bind{\var}{\promise}{\type}}{\kind}}
&=&
\ptarrow{\erase{\type}}{\promise}{\erase{\kind}}
\\

\end{array}
\end{math}
\end{center}
\caption{Dependency erasure.}
\end{figure}


\begin{figure}
\boxed{
\wellsharedterm{\tyenv}{\term}{\pretype}
}
\begin{center}
\begin{mathpar}
\infer{
\wellsharedhead{\tyenv}{\head}{\pused}{\pretype_1}\\\\
\wellsharedspine{\tyenv}{\pretype_1}{\listof\head}{\pretype_2}
}{
\wellsharedterm{\tyenv}{\tapp{\head}{\listof\head}}{\pretype_2}
}

\infer
{
\wellsharedterm{\tenvbind\tyenv{\var}{\promisse}{\type}}
               {\term}
               {\pretype}
}{
\wellsharedterm{\tyenv}
               {\tlam{\bind{\var}{\promisse}{\type}}{\term}}
               {\ptarrow{\erase{\type}}{\promisse}{\pretype}}
}

\infer{
\wellsharedadefs{\tyenv}{\listof\head} \\\\
\wellsharedterm{\tenvcons{\tyenv}{\listof\head}}{\term}{\sigma}
}{
\wellsharedterm{\tyenv}{\tdef{\listof\head}{\term}}{\sigma}
}

\infer{
\openscope{\tyenv}{\tenvlookup{\tyenv}{\var}}{\listof\head}\\\\
\wellsharedterm{\tenvcons{\tyenv}{\listof\head}}{\term}{\sigma}
}{
\wellsharedterm{\tyenv}{\topen{\term}{\var}}{\sigma}
}
\end{mathpar}
\end{center}


\boxed{
\wellsharedhead{\tyenv}{\head}{\promisse}{\pretype}
}
\begin{center}
\begin{mathpar}
\infer{
\tenvlookup{\tyenv}{\var} = \bind{\var}{\promisse}{\type}
}{
\wellsharedhead{\tyenv}{\var}{\promisse}{\erase{\type}}
}

%
% J'ai l'impression que l'on peut sans risque affirmer
% qu'une définition est toujours utilisable. En effet,
% si on avait un alias de la forme :
%       fun (x :_unused A). (y = x) in y
% alors la définition [y = x] serait mal typée.
% Cela briserait l'invariant que les environnements 
% ne contiennent que des définitions bien typées par
% ce système. 
\infer{
\tenvlookup{\tyenv}{\var} = (\oftype{\term}{\type})
}{
\wellsharedhead{\tyenv}{\var}{\pused}{\erase{\type}}
}

\infer{
}{
\wellsharedhead{\tyenv}{\constant}{\pused}{\erase{\type}}
}
\end{mathpar}
\end{center}

\boxed{
\wellsharedspine{\tyenv}{\pretype}{\listof\head}{\pretype}
}
\begin{center}
\begin{mathpar}
\infer{
}{
\wellsharedspine{\tyenv}{\pretype}{\emptylist}{\pretype}
}

\infer{
\wellsharedhead{\tyenv}{\head}{\promisse_2}{\pretype_1'} \\\\
\promisse_1 \leq \promisse_2 \\
\subpretype{\pretype_1'}{\pretype_1}
}{
\wellsharedspine{\tyenv}{\ptarrow{\pretype_1}{\promisse_1}{\pretype_2}}
                {\head\listof\head}
                {\pretype_3}
}
\end{mathpar}
\end{center}

\boxed{
\wellsharedadefs{\tyenv}{\listof\adef}
}
\begin{center}
\begin{mathpar}
\infer{
}{
\wellsharedadefs{\tyenv}{\emptylist}
}

\infer{
\wellsharedterm{\tyenv}{\term}{\erase{\type}}\\
\wellsharedadefs{\tenvdef{\tyenv}{\var}{\type}{\term}}{\listof\adef}
}{
\wellsharedadefs{\tyenv}{\xadef{\var}{\type}{\term}\listof\adef}
}
\end{mathpar}
\end{center}

\boxed{
\wellsharedtype{\tyenv}{\type}
}
\begin{center}
\begin{mathpar}
\infer{
\wellsharedtype{\tenvcons\tyenv\binding}{\type}
}{
\wellsharedtype{\tyenv}{\typrod{\binding}{\type}}
}

\infer{
\wellsharedadefs{\tyenv}{\listof\adef} \\\\
\wellsharedtype{\tenvcons{\tyenv}{\listof\adef}}{\type}
}{
\wellsharedtype{\tyenv}{\tydef{\listof\adef}{\type}}
}

\infer{
\openscope{\tyenv}{\tenvlookup{\tyenv}{\var}}{\listof\head}\\\\
\wellsharedtype{\tenvcons\tyenv{\listof\head}}{\type}
}{
\wellsharedtype{\tyenv}{\tyopen{\type}{\var}}
}

\infer{
% \signature\tyconstant = \kind
\wellsharedkindspine{\tyenv}{\erase\kind}{\listof\head}
}{
\wellsharedtype{\tyenv}{\tyapp{\tyconstant}{\listof\head}}
}
\end{mathpar}
\end{center}

\boxed{
\wellsharedkindspine{\tyenv}{\prekind}{\listof\adef}
}
\begin{center}
\begin{mathpar}
\infer{
\wellsharedhead{\tyenv}{\head}{\promisse_2}{\pretype'} \\\\
\promisse_1 \leq \promisse_2 \\
\subpretype{\pretype_1'}{\pretype_1}
}{
\wellsharedkindspine{\tyenv}
                    {\pkarrow{\pretype}{\promisse_1}{\prekind}}
                    {\head\listof\head}
}

\infer{
}{
\wellsharedkindspine{\tyenv}{\pkconst}{\emptylist}
}
\end{mathpar}
\end{center}

\boxed{
\subpretype{\pretype}{\pretype}
}
\begin{center}
\begin{mathpar}
\infer{
}{
\subpretype{\ptconst}{\ptconst}
}

\infer{
\promisse \leq \promisse' \\
\subpretype{\pretype_1'}{\pretype_1} \\
\subpretype{\pretype_2}{\pretype_2'} \\
}{
\subpretype{\ptarrow{\pretype_1}{\promisse}{\pretype_2}} 
           {\ptarrow{\pretype_1'}{\promisse'}{\pretype_2'}}
}
% FIXME: Rajouter transitivite et reflexivite?
\end{mathpar}
\end{center}

\caption{Well-formed sharing.}
\label{fig:well-shared}
\end{figure}


% FIXME: Expliquer les regles.

\begin{theorem}
If $\wellsharedterm{\Gamma}{\term}{\pretype}$ holds, then 
$\openscope{\Gamma}{\term}{\listof\adef}$ holds. 
\end{theorem}

\subsubsection{From Spine Form LF to {\system}, and back}

% FIXME: Ici, dire que toute forme beta-normale bien typée du Spine Calcul
% FIXME: se traduit vers un objet de notre system bien formé pour les
% FIXME: règles précédentes. 

% FIXME: Donner la procedure! 

% FIXME: Réciproquement, la variante de l'expansion où l'on fait une béta-
% FIXME: réduction après chaque expansion produit une forme canonique.

% FIXME: En d'autres termes, ce langage capture les formes beta-normales
% FIXME: du SC modulo du partage. On fait une distinction entre ``calcul''
% FIXME: et ``partage'' : on ne peut pas calculer statiquement le nombre
% FIXME: de beta-reduction à effectuer pour obtenir la forme normale d'un
% FIXME: calcul tandis que l'on peut le faire en présence de partage. 
% FIXME: Cette quantité est sans doute une bonne mesure du partage.

% FIXME: Enoncer les theoremes de correction/completude.

\subsection{Commit as a type checking algorithm}

% FIXME: Indiquer quelque part qu'en pratique, on infere les annotations
% FIXME: de type des definitions mais qu'on les suppose donnees ici pour
% FIXME: simplifier la presentation. 


\subsubsection{Typed repositories, and their operations}

\paragraph{checkout}

\paragraph{commit}
\begin{figure}
\boxedtitle{\tcterm{\tyenv}{\term}{\type}}
\begin{center}
\begin{mathpar}
\infer{
\tcterm{\tenvcons{\tyenv}{\binding}}{\term}{\type}
}{
\tcterm{\tyenv}{\tlam{\binding}{\term}}{\typrod{\binding}{\type}}
}

\infer{
\openscope{\tyenv}{\tenvlookup{\tyenv}{\var}}{\adefs} \\
\tcterm{\tyenv\adefs}{\term}{\type}
}{
\tcterm{\tyenv}{\topen{\term}{\var}}{\tyopen{\type}{\var}}
}
\\
\infer{
\tcdefs{\tyenv}{\adefs} \\\\
\tcterm{\tenvcons{\tyenv}{\adefs}}{\term}{\type}
}{
\tcterm{\tyenv}{\tdef{\adefs}{\term}}{\tydef{\adefs}\type}
}

\infer{
\distinctheads{\tyenv}{\spine}\\\\
\tcspine{\tyenv}{\tenvlookuptype{\tyenv}{\head}}{\spine}{\type}
}{
\tcterm{\tyenv}{\tapp{\var}{\spine}}{\type}
}

\infer{
\tcspine{\tyenv}{\tenvlookuptype{\tyenv}{\head}}{\spine}{\type}
}{
\tcterm{\tyenv}{\tapp{\constant}{\spine}}{\type}
}
\end{mathpar}
\end{center}
\boxedtitle{\distinctheads{\tyenv}{\spine}}
\begin{center}
\begin{mathpar}
\infer{
}{
\distinctheads{\tyenv}{\emptylist}
}

\infer{
\forall \head', \spine = \spine_1 \head' \spine_2 \Rightarrow \head \not\equiv \head' \\
\distinctheads{\tyenv}{\spine}
}{
\distinctheads{\tyenv}{\head\spine}
}
\end{mathpar}
\end{center}
\boxedtitle{\tcspine{\tyenv}{\type}{\spine}{\type}}
\begin{center}
\begin{mathpar}
\infer{
\equaltype{\tyenv}{\type}{\tenvlookuptype{\tyenv}{\head}} \\\\
\tcspine{\tenvdef{\tyenv}{\var}{\head}{\type}}
        {\typebis}
        {\spine}
        {\typebis'}
}{
\tcspine{\tyenv}
        {\typrod{\bind{\var}{\promise}{\type}}{\typebis}}
        {\head\spine}
        {\typebis'}
}

\infer{
\openscope{\tyenv}{\tenvlookup{\tyenv}{\var}}{\adefs} \\\\
\tcspine{\tenvcons{\tyenv}{\adefs}}
        {\type}
        {\spine}
        {\typebis}
}{
\tcspine{\tyenv}
        {\topen{\type}{\var}}
        {\spine}
        {\tydef{\adefs}\typebis}
}

\infer{
\tcspine{\tyenv\adefs}
        {\type}
        {\spine}
        {\typebis}
}{
\tcspine{\tyenv}
        {\tydef{\adefs}{\type}}
        {\spine}
        {\tydef{\adefs}{\typebis}}
}

\infer{
}{
\tcspine{\tyenv}
        {\type}
        {\emptylist}
        {\type}
}
\end{mathpar}
\end{center}

\boxedtitle{\tcdefs{\tyenv}{\adefs}}
\begin{center}
\begin{mathpar}
\infer{
}{
\tcdefs{\tyenv}{\emptylist}
}

\infer{
\tcterm{\tyenv}{\term}{\typebis} \\
\equaltype{\tyenv}{\type}{\typebis} \\
\tcdefs{\tyenv\xadef{\var}{\term}{\type}}{\adefs}
}{
\tcdefs{\tyenv}{\xadef{\var}{\term}{\type}\adefs}
}
\end{mathpar}
\end{center}

% Kind? Signature? Environment?
\caption{Type checking algorithm.}
\end{figure}


\paragraph{reshare}

\subsubsection{Type-checking patchs}

% FIXME: Ici, explication des regles de types. 

\subsubsection{Sharing-preserving equality decision procedure}

\paragraph{Lazy weak head normal forms}
\begin{figure}
\boxedtitle{
\whnf{\tyenv}{\term}{\type}{\spine}{\adefs}{\term}
}
\begin{center}
\begin{mathpar}
\inferrule*[left=N-Beta]{
\whnf{\tenvcons{\tyenv}{\xadef{\var}{\head'}{\promise}{\type}}}
     {\term}
     {\typebis}
     {{\spine}'}
     {\adefs}
     {\term}
}{
\whnf{\tyenv}
     {\tlam{\bind\var\promise\type}{\term}}
     {\typrod{\bind{\var}{\promise}{\type}}{\typebis}}
     {\head'{\spine}'}
     {\xadef{\var}{\head'}{\promise}{\type}\adefs}
     {\term}
}

\inferrule*[left=N-Push]{
\tcterm{\tyenv}{\head}{\pany}{\typebis'} \\
\tcterm{\tyenv}{\head'}{\promise'}{\type'} \\
\promise' \le \promise\\
% FIXME: \equaltype{\tyenv}{\type}{\type'} Redondante? \\
\whnf{\tenvcons{\tyenv}{\xadef{\var}{\head'}{\promise}{\type}}}
     {\tapp{\head}{\concat{\spine}{\var}}}
     {\typebis}
     {{\spine}'}
     {\adefs}
     {\term}
}{
\whnf{\tyenv}
     {\tapp{\head}{\spine}}
     {\typrod{\bind{\var}{\promise}{\type}}{\typebis}}
     {\head'{\spine}'}
     {\xadef{\var}{\head'}{\promise}{\type}\adefs}
     {\term}
}

\inferrule*[left=N-Expand]{
\wellsharedhead{\tyenv}{\head'}{\promise'}{\pretype} \\
\tenvlookup{\tyenv}{\var} = (\oftype{\term}{\pany}{\type}) \\\\
\whnf{\tyenv}
     {\term}
     {\type}
     {\spine \head' {\spine}'}
     {\adefs}
     {\term}
}{
\whnf{\tyenv}
     {\tapp{\var}{\spine}}
     {\typebis}
     {\head'{\spine}'}
     {\adefs}
     {\term}
}

\inferrule*[left=N-Finish-App]{
}{
\whnf{\tyenv}
     {\tapp{\head}{\spine}}
     {\type}
     {\emptylist}
     {\emptylist}
     {\tapp{\head}{\spine}}
}

\inferrule*[left=N-Finish-Abs]{
}{
\whnf{\tyenv}
     {\tlam{\binding}{\term}}
     {\type}
     {\emptylist}
     {\emptylist}
     {\tlam{\binding}{\term}}
}

\inferrule*[left=N-Open]{
\openscope{\tyenv}{\tenvlookup{\tyenv}{\var}}{\adefs} \\
\whnf{\tenvcons{\tyenv}{\adefs}}
     {\term}
     {\type}
     {\spine}
     {{\adefs}'}
     {\term}
}{
\whnf{\tyenv}
     {\topen{\term}{\var}}
     {\type}
     {\spine}
     {\adefs{\adefs}'}
     {\term}
}

\inferrule*[left=N-Def]{
\whnf{\tenvcons{\tyenv}{\adefs}}
     {\term}
     {\type}
     {\spine}
     {{\adefs}'}
     {\term}
}{
\whnf{\tyenv}
     {\tdef{\adefs}{\term}}
     {\type}
     {\spine}
     {\adefs{\adefs}'}
     {\term}
}

\inferrule*[left=N-Open-Type]{
\openscope{\tyenv}{\tenvlookup{\tyenv}{\var}}{\adefs} \\
\whnf{\tenvcons{\tyenv}{\adefs}}
     {\term}
     {\type}
     {\spine}
     {{\adefs}'}
     {\term}
}{
\whnf{\tyenv}
     {\term}
     {\topen\type\var}
     {\spine}
     {\adefs{\adefs}'}
     {\term}
}

\inferrule*[left=N-Def-Type]{
\whnf{\tenvcons{\tyenv}{\adefs}}
     {\term}
     {\type}
     {\spine}
     {{\adefs}'}
     {\term}
}{
\whnf{\tyenv}
     {\term}
     {\tdef\adefs\type}
     {\spine}
     {\adefs{\adefs}'}
     {\term}
}
\end{mathpar}
\end{center}
\caption{Weak head normalization with precautious expansion of definitions.}
\end{figure}


\paragraph{Equality decision procedure}
\begin{figure}
\boxed{
\equalterm{\tyenv}{\term}{\term}{\type}
}
\begin{center}
\begin{mathpar}
\inferrule*[left=Eq-whnf]{
\whnf{\tyenv}{\term}{\type}{\emptylist}{\adefs}{\term'}\\
\whnf{\tyenv}{\termbis}{\type}{\emptylist}{{\adefs}'}{\termbis'}\\\\
\equalwhnfterm{\tenvcons{\tenvcons{\tyenv}{\adefs}}{{\adefs}'}}
              {\term'}
              {\termbis'}
              {\type}
}{
\equalterm{\tyenv}{\term}{\termbis}{\type}
}
\end{mathpar}
\end{center}

\boxed{
\equalwhnfterm{\tyenv}{\term}{\term}{\type}
}
\begin{center}
\begin{mathpar}
\inferrule*[left=Eq-Inj-Head]{
\equalspine{\tyenv}{\tenvlookuptype{\tyenv}{\head}}{\spine}{{\spine}'}
}{
\equalwhnfterm{\tyenv}{\tapp{\head}{\spine}}{\tapp{\head}{{\spine}'}}
              {\type}
}

\inferrule*[left=Eq-Diff-Head]{
\whnfbis{\tyenv}
     {\tenvlookup{\tyenv}{\var}}
     {\spine}
     {\adefs}{\term'}\\\\
\whnfbis{\tyenv}
     {\tenvlookup{\tyenv}{\var'}}
     {{\spine}'}{{\adefs}'}{\termbis'}\\\\
\equalwhnfterm{\tenvcons{\tenvcons{\tyenv}{\adefs}}{{\adefs}'}}
              {\term'}
              {\termbis'}
              {\type}
}{
\equalwhnfterm{\tyenv}
              {\tapp{\var}{\spine}}
              {\tapp{\var'}{{\spine}'}}
              {\type}
}

\inferrule*[left=Eq-Abs-App]{
\equalterm{\tenvbind{\tyenv}{\var}{\promise}{\type}}
          {\term}
          {\tapp{\head}{\concat{\spine}{\var}}}
          {\type}
}{
\equalwhnfterm{\tyenv}
              {\tlam{\bind{\var}{\promise}{\type}}{\term}}
              {\tapp{\head}{\spine}}
              {\typrod{\bind{\var}{\promise}{\type}}\type}
}

% FIXME: Meme \promise?
\inferrule*[left=Eq-Abs-Abs]{
\equalterm{\tenvbind{\tyenv}{\var}{\promise}{\type}}
          {\term}
          {\termbis}
          {\type}
}{
\equalwhnfterm{\tyenv}
              {\tlam{\bind{\var}{\promise}{\type}}{\term}}
              {\tlam{\bind{\var}{\promise}{\type}}{\termbis}}
              {\typrod{\bind{\var}{\promise}{\type}}\type}
}

% FIXME: Manque les cas ``type = \tyopen{}{}'' et ``type = \tydef{}{}''. 
% FIXME: Manque aussi quelque cas symetrique ... 
% FIXME: On omet?
\end{mathpar}
\end{center}

\boxed{
\equaltype{\tyenv}{\type}{\type}
}
\begin{center}
\begin{mathpar}
% FIXME: Il faut argumenter ici que le choix arbitraire de \promise est correct. 
\inferrule*[left=Eq-Prod-Type]{
\equaltype{\tyenv}{\type}{\type'} \\
\equaltype{\tenvbind{\tyenv}{\var}{\promise}{\type}}
          {\typebis}
          {\typebis'}
}{
\equaltype{\tyenv}
          {\typrod{\bind{\var}{\promise}{\type}}{\typebis}}
          {\typrod{\bind{\var}{\promise'}{\type'}}{\typebis'}}
}

\inferrule*[left=Eq-App-Type]{
% FIXME: Refer to signature for constant!
\equalspine{\tyenv}{\tenvlookuptype{\tyenv}{\tyconstant}}{\spine}{{\spine}'}
}{
\equaltype{\tyenv}
          {\tyapp{\tyconstant}{\spine}}
          {\tyapp{\tyconstant}{{\spine}'}}
          {}
}
% FIXME: Manque les cas ``type = \tyopen{}{}'' et ``type = \tydef{}{}''. 
% FIXME: Manque aussi quelque cas symetrique ... 
% FIXME: On omet?
\end{mathpar}
\end{center}

\boxed{
\equalspine{\tyenv}{\type}{\spine}{\spine}
}
\begin{center}
\begin{mathpar}
\inferrule*[left=Eq-Empty-Spine]{
}{
\equalspine{\tyenv}{\type}{\emptylist}{\emptylist}
}

\inferrule*[left=Eq-Equal-Spine-Head]{
\equalspine{\tyenv}
           {\typebis}
           {{\spine}}
           {{\spine}'}
}{
\equalspine{\tyenv}
           {\typrod{\bind{\var}{\promise}{\type}}{\typebis}}
           {\conslist\head{\spine}}
           {\conslist\head{\spine}'}
}

\inferrule*[left=Eq-Diff-Spine-Head]{
\equalterm{\tyenv}{\tenvlookup{\tyenv}{\var}}{\tenvlookup{\tyenv}{\var'}}{\type}\\
\equalspine{\tenvdef{\tyenv}{\var}{\var'}{\promise}{\type}}
           {\typebis}
           {{\spine}}
           {{\spine}'}
}{
\equalspine{\tyenv}
           {\typrod{\bind{\var}{\promise}{\type}}{\typebis}}
           {\var{\spine}}
           {\var'{\spine}'}
}
\end{mathpar}
\end{center}

% FIXME: Kind equality? 
\caption{Decision procedure for equality. (Some obvious cases are omitted.)}
\end{figure}


\subsubsection{Soundness and Completeness}

\begin{definition}[A-injectivity]
A definition $\adef \equiv \xadef{\var}{\term}{\type}$ is A-injective
in $\tyenv$ iff for all $\spine_1 \spine_2$ such that
$\tcterm{\tenvcons{\tyenv}\adef}{\tapp\var{\spine_1}}{\typebis}$ holds
$\tcterm{\tenvcons{\tyenv}\adef}{\tapp\var{\spine_2}}{\typebis}$
holds, and
$\equalterm{\tenvcons{\tyenv}{\adef}}{\tapp{\var}{\spine_1}}
           {\tapp{\var}{\spine_2}}
           {\typebis}$
holds, then $\equalspine{\tyenv}{\type}{\spine_1}{\spine_2}$ holds. 
\end{definition}

\begin{theorem}
If $\tcterm{\tyenv}{\term}{\type}$ holds then $\xadef{\var}{\term}{\type}$ is
$\type$-injective in~$\tyenv$.
\end{theorem}

\begin{theorem}[Soundness]\ \\[-1em]
\begin{itemize}
\item  If $\tcterm{\tyenv}{\term}{\type}$ holds then 
$\tcterm{\interprettyenv{\tyenv}}{\interpret{\term}{\tyenv}}{\interpret{\type}{\tyenv}}$
holds in the Spine~Calculus. 
\item If $\equalterm{\tyenv}{\term}{\termbis}{\type}$ holds then
$\equalterm{\interprettyenv{\tyenv}}{\interpret{\term}{\tyenv}}
           {\interpret{\termbis}{\tyenv}}{\interpret{\type}{\tyenv}}$
holds in the Spine~Calculus. 
\item \ldots
\end{itemize}
\end{theorem}

\begin{theorem}[Completeness]\ \\[-1em]
\begin{itemize}
\item  If $\tcterm{\interprettyenv{\tyenv}}{\interpret{\term}{\tyenv}}{\interpret{\type}{\tyenv}}$ 
holds in the Spine~Calculus then $\tcterm{\tyenv}{\term}{\type}$ holds.
\item If
$\equalterm{\interprettyenv{\tyenv}}{\interpret{\term}{\tyenv}}
           {\interpret{\termbis}{\tyenv}}{\interpret{\type}{\tyenv}}$ holds in the Spine~Calculus
then $\equalterm{\tyenv}{\term}{\termbis}{\type}$ holds.
\item \ldots
\end{itemize}
\end{theorem}

\subsection{Safe incremental type checking using {\system}}

\subsubsection{Architecture}

\subsubsection{A formal definition of incremental type checking}

% Application
\section{Application to simply typed $\lambda$-calculus}

\section{Related work}

\section{Future work}

\section{Conclusion}

\appendix
\section{Appendix Title}

This is the text of the appendix, if you need one.

\acks

Acknowledgments, if needed.

% We recommend abbrvnat bibliography style.

\bibliographystyle{abbrvnat}

% The bibliography should be embedded for final submission.

\begin{thebibliography}{}
\softraggedright

\bibitem[Smith et~al.(2009)Smith, Jones]{smith02}
P. Q. Smith, and X. Y. Jones. ...reference text...

\end{thebibliography}

\end{document}
