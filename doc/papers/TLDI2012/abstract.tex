\documentclass[9pt,authoryear]{sigplanconf}

\usepackage{ntheorem}
\usepackage{amsmath}
\usepackage{amssymb}
\newtheorem{theorem}{Theorem}
\newtheorem{definition}{Definition}
\usepackage{mathpartir}
\usepackage{macros}

\begin{document}

\conferenceinfo{TLDI '12}{January 28 2012, Philadelphia, USA.}
\copyrightyear{2012} 
\copyrightdata{[to be supplied]} 

\titlebanner{banner above paper title}        % These are ignored unless
\preprintfooter{short description of paper}   % 'preprint' option specified.

\title{Incremental Type Checking}
%\subtitle{Subtitle Text, if any}

\authorinfo{Matthias Puech}
           {University of Bologna, University Paris Diderot}
           {puech@cs.unibo.it}
\authorinfo{Yann R\'egis-Gianas}
           {University Paris Diderot, CNRS, and INRIA}
           {yrg@pps.jussieu.fr}

\maketitle

\begin{abstract}
  We study the problem of verifying the well-typing of terms, not in a
  batch fashion, as it is usually the case for typed languages, but
  incrementally, that is by sequentially modifying a term, and
  re-verifying each time only a smaller amount of information.
\end{abstract}

\category{CR-number}{subcategory}{third-level}

\terms
term1, term2

\keywords
keyword1, keyword2

\section{Introduction}

As programs grow and type systems become more involved, writing a
correct program in one shot becomes quite difficult. On the other
hand, writing a program in many correct steps can be tedious because
generally, the verification tool rechecks the entire development at
each step. This second option is the usual practice when the time for
verification is negligible with respect to the time to make the
change, but this is less and less the case, especially when the
language in question contains proof aspects, and verification involves
proof search. Some mechanisms already exist to cope with the
incrementality of proofs or program development: separate compilation,
interactive toplevel with undo, tactic languages; they all provide in
different ways a rough approximation of the process of modifying and
checking incrementally a large term.

We propose here an architecture for a generic incremental
type-checker, a data structure for repositories of typed proofs and a
language for describing proof deltas. It is based on the simple idea
of sharing common subterms to avoid rechecking and exploits encoding a
derivation in a metalanguage. This way, given a signature declaring
the typing rules and an (untrusted) typing algorithm for my language
of choice, I get an incremental type-checker for that language. The
metalanguage approach gives us the ability to encode all the
aforementioned usual incrementality mechanisms, and more, making our
system akin to a typed version control system.

\section{Sharing-based incrementality}

As a first example, let us consider a purposedly simplistic sorted
language of boolean and arithmetic expressions:
$$ e, e' \gequal n \gor e + e' \gor e \land e' \gor e \leq e' $$
The algorithm to determine in a batch fashion wether the term
$$ e_1 = (1 + 3 \leq 2 + 4) \land (8 \leq 3) $$ is well-sorted is
trivial (we don't care about its evaluation here, just its
well-sortedness). But what if I then change subterm $2+4$ in $e_1$
into $7 \leq 2+4$, to obtain $e_2$? Clearly, it should be verified
that context $7\leq []$ is well-sorted (it is), that $2+4$ ``fits''
into its hole (it does), that the whole expression ``fits'' into its
new context $(1+3\leq [])\land(8\leq 3)$ (it does not); but the other,
unchanged subterms need not be verified again. To achieve this
incremental verification, the system would have to ``remember'' the
states of the verifier in some way.

If only we had \emph{names} (memory addresses, hashes) for enough
subterms of our initial term, $$ e_1 = (\overbrace{1 + 3}^\mmeta \leq
\overbrace{2+4}^\mmmeta) \land \overbrace{(8 \leq 3)}^\mmmmeta\text{\
  ,} $$ we could express concisely the change as a
\emph{delta} $$\delta_1 = (\mmeta\leq(7\leq \mmmeta))\land \mmmmeta
\text{\ ,}$$ using these names to refer to unchanged subterms. If only
we had \emph{annotated} our initial term with the states of the
verifier,
$$e_1 = (\overbrace{1 + 3}^{\mmeta:\nat} \leq \overbrace{2 +
  4}^{\mmmeta:\nat}) \land \overbrace{(8 \leq
  3)}^{\mmmmeta:\bool}\text{\ ,} $$ we would have a simple process to
verify $e_2$ taking advantage of $e_1$'s derivation, in
$O(|\delta_2|)$: verify $\delta_1$ as a term, retrieving the sort of
names from a stored map. This suggests a data structure for a
\emph{repository} of named, annotated, verified subexpressions: a
monotonously growing map, from names, or \emph{metavariables}, to terms
and types, together with a \emph{head} metavariable identifying the
top of the expression: $$\mr \gequal X, \Delta \quad\text{where}\quad
\Delta : (\mmeta\mapsto M : A) \text{\ .}$$

\section{The metalanguage}

What language are terms $M$ and types $A$ written into? Terms should
encode our expressions with metavariables, and types $A$ should encode
the whole state of the batch verifier (here the sort). An obvious
choice is to take $M\gequal(e \text{ with metavariables})$ and $A\gequal
\bool\gor\nat$, but switching to another language, we'd have to
redefine another repository language. Another more modular choice for
this is the \emph{metalanguage} \emph{LF} \cite{harper1993framework}: it allows
to specify syntax and rules of an object language as a
\emph{signature} $\Sigma$, and check terms against this signature with
a generic algorithm. We'll use an increasing fragment of it. The
so-called \emph{intrinsic} style of LF signature for our expression
language is:
\begin{align*}
  tp &: \type,\quad nat : tp,\quad bool : tp,\quad exp : tp\to nat,\\
  atom &: \nat\to exp\ nat,\\
  plus &: exp\ nat \to exp\ nat \to exp\ nat,\\
  and &: exp\ bool\to exp\ bool\to exp\ bool,\\
  leq &: exp\ nat\to exp\ nat\to exp\ bool
\end{align*}
In this style, both the encoding of an expression and its sort are
\emph{terms} in the metalanguage, but the sort appears in the
\emph{type} of the encoded expression. As an example, the repository
associated with expression $e_1$ is
\begin{align*}
  T, \left(
  \begin{array}{l}
  X \mapsto plus\ 1\ 3 : exp\ nat \\
  Y \mapsto plus\ 2\ 4 : exp\ nat \\
  Z \mapsto leq\ 8\ 3 : exp\ bool \\
  T \mapsto and\ (leq\ X\ Y)\ Z : exp\ bool \\
\end{array}\right)
\end{align*}

The dependent nature of types in LF allows to express more complex
languages. We can for example add functions, applications and
variables to our expressions in a purely first-order style (using de
Bruijn indices for variables) if we annotate them not only with sorts
but with an environment of free variables:
\begin{align*}
  exp &: env\to tp\to \type,\\
  atom &: \prd E {env} \nat\to exp\ E\ nat,\\
  var &: \prd E {env} \prd A {tp} var\ E\ A \to exp\ E\ A,\\
  leq &: \prd E {env} exp\ e\ nat\to exp\ E\ nat\to exp\ E\ bool, \\
  lam &: \prd E {env} \prd {A, B} {tp} exp\ (cons\ A\ E)\ B \to
  exp\ E\ (arr\ A\ B)\\
  \ldots
\end{align*}

The encoded expressions are however very verbose: each term
constructor takes as argument all these annotations. We can
nonetheless make these information \emph{implicit} in terms (but
explicit in types) and let a reconstruction algorithm infer them, as
in \cite{necula1997efficient}. This reconstruction is
language-dependent, user-provided but does not impair the safety of
the system for the whole term is still checked afterwards.

There is plenty of room for improvement here: one could for instance
wish to make use of admissible rules to increase sharing (e.g. adding
a $lam$ constructor at the top of a term requires to rewrite the whole
term, or better, apply weakening on it); this supposes to support
computations on derivation. Besides, LF promotes the use of
lambda-tree syntax to represent binders; it would require a mechanism
toimplicitly manage and share the environment in our metalanguage.

\section{Expressivity}

Aside from enabling to encode a large class of languages generically,
the metalanguage approach allows to express incrementality features
usually implemented in an ad-hoc manner, simply by adding new
constants to the signature.

Suppose we want to implement an \emph{undo system}, storing successive
versions of a closed expression of sort $bool$ and able to rollback to
a previous version. We add constants
\begin{align*}
  &version : \type,\quad vnil : version,\\
  &vcons : exp\ nil\ bool \to version \to version
\end{align*}
to the signature. The empty repository is now represented as
$vnil$. Each time we have pushed a full expression $M$, and if $S$ was
the previous head (a $version$ called its \emph{ancestor}), we push
$vcons\ M\ S$. This gives us a data structure for an undo stack, and a
\emph{commit} algorithm. But the sharing inherent to our repositories
lets us actually represent \emph{trees} of versions, by sharing common
stack tails, each list head being a \emph{branch}. Reconciling two
branches' changes into a unique head is called \emph{merging} in
version control system's terminology: a merge is a version with
several ancestors. We can represent merges by revising our previous
addition to the signature into
\begin{align*}
  &version : \type,\quad ancestors : \type,\quad anil : ancestors,\\
  &acons : version \to ancestors \to ancestors,\\
  &vcons : exp\ nil\ bool \to ancestors \to version
\end{align*}
This defines a data structure to represent (acyclic) \emph{graphs} of
versions; it is the exact data structure of repository used by version
control systems \textsf{Git}, \textsf{Monotone} and \textsf{Mercurial}
(see \cite{baudis2009current}).

While our system is based on \emph{bottom-up} term construction, we
can encode \emph{top-down} construction common to some programming
environments (e.g. \textsf{Agda}) and tactic-based proof assistants
(\textsf{Coq}). The user constructs terms by
successively filling \emph{holes} with terms containing other
holes. To add (linear) holes to our expressions, add constant
$$ hole : \prd E {env} \prd A {tp} exp\ E\ A $$
to the signature. To instantiate a hole with an expression, commit the
substituted term preserving sharing of subterms.

\section{Architecture}

The system follows a layered architecture.

The \emph{kernel} is the component in charge of verifying terms
against a signature and a repository, and updating this repository. It
supports two basic operations:
\begin{itemize}
\item $\pfunction{push}{\Sigma}{\mr,M}$ checks $M$ against $\Sigma$
  in $\mr$, synthetizes its type $A$, chooses a fresh metavariable
  $\mmeta$ for $M$ and returns $\mr[\mmeta\mapsto M:A]$ and $\mmeta$.
\item $\pfunction{pull}\Sigma{\mr, \mmeta}$ returns the term $M$
  associated with $\mmeta$ in $\mr$ recursively: all metavariables are
  unfolded to their definitions.
\end{itemize}

The \emph{slicer} is the component in charge of slicing a term $M$
into many terms, pushing them to the repository to enable future
sharing, and adding version markers. It supports operations:
\begin{itemize}
\item $\pfunction{commit}\Sigma{\mr, M}$ pushes $vcons\ M\ (acons\ \mmeta\
  anil)$ to $\mr$ in several $\function{push}{}$ operations, where
  $\mmeta$ is the current head.
\item $\pfunction{merge}\Sigma{\mr, M, \mmmeta}$ pushes $vcons\ M\
  (acons\ \mmeta\ (acons\ \mmmeta\ anil))$ to $\mr$. Note that it
  doesn't actually perform the merge, it simply commits a previously
  computed merge node with value $M$.
\end{itemize}

The \emph{reconstructor} performs the reconstruction of the derivation
(an $M$) from the initial expression (an $e$), given the
derivations for its metavariables (an $\mr$), and the expected type
($exp\ nil\ bool$), and commits $M$.

Finally, the \emph{compressor} computes a delta $e'$ from a
metavariable-free expression $e$ by recognizing equal subterms. This
can be achieved by \emph{hash-consing}.

%%%%%%%%%%%%%%%%%%%%%%%%%%%%%%%%%%%%%%%%%%%%%%%%%%%%%%%%%%%%%%%%%%%

\bibliographystyle{abbrvnat}
\bibliography{../../english.bib}


% \begin{thebibliography}{}
% \softraggedright
% \bibitem[Smith et~al.(2009)Smith, Jones]{smith02}
% P. Q. Smith, and X. Y. Jones. ...reference text...
% \end{thebibliography}

\end{document}
