\documentclass[9pt]{sigplanconf}

\usepackage{ntheorem}
\usepackage{amsmath}
\usepackage{amssymb}
\newtheorem{theorem}{Theorem}
\newtheorem{definition}{Definition}
\usepackage{macros}

\begin{document}

\conferenceinfo{LFMTP '12}{September 9, 2012 -- Copenhagen, Denmark}
\copyrightyear{2012}
\copyrightdata{[to be supplied]}

\titlebanner{banner above paper title}        % These are ignored unless
\preprintfooter{short description of paper}   % 'preprint' option specified.

\title{Certified incremental program transformations}
%\subtitle{Subtitle Text, if any}

\authorinfo{Matthias Puech}
           {Department of Comp. Sci., Univ. of Bologna,\\
             PPS, Team $\pi r^2$ (Univ. Paris Diderot, CNRS, INRIA)}
           {puech@cs.unibo.it}

\authorinfo{Yann R\'egis-Gianas}
           {PPS, Team $\pi r^2$ (Univ. Paris Diderot, CNRS, INRIA)}
           {yrg@pps.jussieu.fr}

\maketitle

\begin{abstract}
\end{abstract}

\category{D.3.3}{Language Constructs and Features}{Data types and
  structures} \category{F.3.1}{Logics and Meanings of
  Programs}{Specifying and Verifying and Reasoning about
  Programs --- Logics of programs}

\terms
Theory, Languages

\keywords
incrementality, type checking, logical framework, version control

%%%%%%%%%%%%%%%%%%%%%%%%%%%%%%%%%%%%%%%%%%%%%%%%%%%%%%%%%%%%%%%%%%%%%%%%%%%%%%%%

\section{Introduction}

\section{The calculus}

\subsection{Core calculus}

\paragraph{Syntax}

We defined the syntax of canonical LF (TODO cite) enriched with
metavariables and substitutions. Canonicity against $\beta$-reduction
is enforced in the syntax, but we think of them also as $\eta$-long;
it will be enforced by typing.

\begin{align*}
  K &\gequal
  \prd x A K \gor
  \type &
  \text{Kind}\\
  A &\gequal
  \prd x A A \gor
  \app {\cst a} S &
  \text{Type family}\\
  M &\gequal
  \lam x M \gor
  \app H S \gor
  \smeta X \sigma &
  \text{Object}\\
  H &\gequal
  \var x \gor
  \cst c &
  \text{Head}\\
  S &\gequal
  \spinenil \gor
  \spinecons S M &
  \text{Spine}\\
  \sigma &\gequal
  \msubstnil \gor
  \msubstcons \sigma x M &
  \text{Substitution} \\
  \Gamma &\gequal
  \gnil \gor
  \gcons \Gamma {\tp {\var x} A} &
  \text{Environment} \\
  \Sigma &\gequal
  \gnil \gor
  \gcons \Sigma {\cst c : A} \gor
  \gcons \Sigma {\cst a : K} &
  \text{Signature}
\end{align*}

As usual, we write $\meta X$ for $\smeta X \msubstnil$, $A\to B$ for
$\prd x A B$ when $x\notin \FV{B}$ and adopt list notations for $S,
\sigma, \Gamma, \Sigma$ when convenient.

A \emph{repository} is a finite map from metavariable to judgements of
object well-typing, along with a distinguished metavariable, the
\emph{head}:
$$ \repo R\;:\;(\meta X \mapsto (\Gamma\vdash\tp M A)), \smeta X \sigma $$

\paragraph{Substitution} We overload \emph{substitution} as a partial
operation on terms, given a parallel substitution $\sigma$. Since the
syntax of object is not stable by substitution, it is defined
\emph{hereditarily}: it is mutually recursive with a
\emph{normalization} function $\gcut M S$ which can in turn trigger a
chain of substitutions, leading to a canonical term, a stuck
substitution or divergence. % TODO vraiment?

\begin{align*}
  \gsubst {(\lam x M)} \sigma &= \lam x {\gsubst M \sigma} &
  \text{if $x\notin dom{(\sigma)}$}
  \\
  \gsubst {(\app {\cst c} S)} \sigma &= \app {\cst c} {(\gsubst S
    \sigma)}
  \\
  \gsubst {(\app {\var x} S)} \sigma &=
  \app {\var x} {\gsubst S \sigma} &
  \text{if $x\notin dom{(\sigma)}$}
  \\
  \gsubst {(\app {\var x} S)} \sigma &=
  \gcut M {\gsubst S \sigma} &
  \text{if $\sigma(x) = M$}
  \\
  \gsubst {\smeta X {\sigma}} {\sigma'} &=
  \smeta X {\gcomp\sigma{\sigma'}}
  \\
  \gsubst\spinenil\sigma &=
  \spinenil \\
  \gsubst{\spinecons S M}\sigma &=
  \spinecons{\gsubst S\sigma}{\gsubst M\sigma}
  \\
  \gcut {\lam x M} {\spinecons N S} &=
  \gcut {\gsubst M N} S
  \\
  % TODO cas meta/S ????
  % \gcut {\smeta X\sigma} S &=
  % \smeta X{\gcomp\sigma\sigma'}
  % \\
  \gcut {\app H S} \spinenil &=
  \app H S
  \\
  \gcomp \sigma \msubstnil &=
  \sigma
  \\
  \gcomp \sigma {(\msubstcons{\sigma'} x M)} &=
  \gcomp {\gsubst\sigma {\msubst x M}} \sigma'
  \\
  \gsubst {(\msubstcons{\sigma'} x M)} \sigma &=
  \msubstcons{\gsubst{\sigma'}\sigma} x {\gsubst M \sigma}
  \\
  \gsubst {\msubstnil}{\sigma} &= \msubstnil
\end{align*}

Note that this notion of substitution presupposes $\eta$-long normal
forms: a spine applied to another application $\gcut {\app H S} {S'}$
blocks the substitution.

% TODO terminaison??

%%%%%%%%%%%%%%%%%%%%%%%%%%%%%%%%%%%%%%%%%%%%%%%%%%%%%%%%%%%%%%%%%%%%%%%%%%%%%%%%

\bibliographystyle{abbrvnat}
\bibliography{../../english.bib}


% \begin{thebibliography}{}
% \softraggedright
% \bibitem[Smith et~al.(2009)Smith, Jones]{smith02}
% P. Q. Smith, and X. Y. Jones. ...reference text...
% \end{thebibliography}

\end{document}
