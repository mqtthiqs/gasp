\documentclass[preprint]{sigplanconf}

\usepackage{amsmath}
\usepackage{ucs}
\usepackage[utf8x]{inputenc}
\usepackage[usenames,dvipsnames]{color}
\usepackage[pdftex,backref=page,colorlinks=true]{hyperref}
\usepackage{ucs}
\usepackage{mathpartir}
\usepackage{amsfonts,amsmath,amscd}
\usepackage{stmaryrd}
\usepackage[utf8x]{inputenc}
\usepackage[protrusion=true,expansion=true]{microtype}
\usepackage{listings}

\hypersetup{
  linkcolor=blue,
  citecolor=blue
}

% Various
\newcommand\gray[1]{\textcolor{gray}{#1}}
\newcommand\cit[1]{[\textcolor{blue}{#1}]}
\newcommand\To{\Rightarrow}
\newcommand\too{\longrightarrow}
\newcommand\Too{\Longrightarrow}
\newcommand\nat{\mathbb N}
\newcommand\FV[1]{\mathsf{FV}({#1})}
\newcommand\notinfv[2]{{#1}\notin\FV{#2}}

\newcommand\itplus{\textcolor{green!60!black}{\textbf{\textsf{+}}}}
\newcommand\itminus{\textcolor{red}{\textbf{\textsf{--}}}}

\newenvironment{smallright}{
  \begin{flushright}
    \footnotesize
  }{
  \end{flushright}
}
\newcommand\memph[1]{\boxed{#1}}

% Remarks
\newcommand\rem[2]{\textcolor{Mahogany}{{\textsf{#1}} $\triangleright$
    \textsl{#2}}}
\newcommand\mrem[1]{\rem{M}{#1}}
\newcommand\yrem[1]{\rem{Y}{#1}}

% Constructors
\newcommand\pair[2]{{#1},{#2}}

% Grammars
\newcommand\gor{\ |\ }
\newcommand\gequal{\ ::=\ }

% Variables
\newcommand\var[1]{#1}
\newcommand\meta[1]{\textsc{#1}}
\newcommand\typ[2]{{#1}:{#2}}

\newcommand\mv{x}
\newcommand\mmv{y}
\newcommand\mmmv{z}

\newcommand\mmeta{\meta x}
\newcommand\mmmeta{\meta y}
\newcommand\mmmmeta{\meta z}
\newcommand\mameta[1]{\typ{\mmeta}{#1}}

\newcommand\mco{c}
\newcommand\mmco{c'}
\newcommand\mcf{a}
\newcommand\mmcf{a'}

% \newcommand\setdefaultlang[1]{\providecommand{\thedefaultlang}{#1}}
% \newcommand\thedefaultlang{UNDEFINED}
% \newcommand\m[2][\thedefaultlang]{\m#2#1}

% Terms
\newcommand\postbinder{\cdot}
\newcommand\prd[2]{\Pi{#1}:{#2}\postbinder}
\newcommand\prdi[1]{\forall{#1}\postbinder}
\newcommand\app[1]{{#1}\ }
\newcommand\tlam[2]{\lambda{#1}:{#2}\postbinder}
\newcommand\ulam[1]{\lambda{#1}\postbinder}
\newcommand\letb[2]{\mathsf{let}\ {#1}={#2}\ \mathsf{in}\ }
\newcommand\lam{\tlam}
\newcommand\obox[3]{\{{#1}\}^{#2}_{#3}}

% Sorts
\newcommand\srt[1]{\mathsf{#1}}
\newcommand\type{\srt *}
\newcommand\kind{\srt \Box}

% Sequent-like application
\newcommand\lapp[2]{{#1}[{#2}]}
\newcommand\laapp[3]{{#1}[{#2}]:{#3}}
\newcommand\lnil{\cdot}
\newcommand\lcons[2]{{#1};{#2}}
\newcommand\lsing[1]{{#1}}
\newcommand\lncons[3]{{#1}={#2};{#3}}
\newcommand\lnsing[2]{{#1}={#2}}

% Environments
\newcommand\enil\cdot
\newcommand\eent[1]{[{#1}]}
\newcommand\econs[2]{{#1}\eent{#2}}
\newcommand\esing[1]{\econs{}{#1}}
\newcommand\emerge[2]{{#1}\cdot{#2}}
\newcommand\elook[2]{{#1}({#2})}
\newcommand\elookdef[3]{{#1}({#2}) = {#3}}
\newcommand\elookdecl[3]{{#1}({#2}) : {#3}}
\newcommand\ebind[2]{\econs{#1}{#2}}
\newcommand\ebinddef[3]{\econs{#1}{{#2}={#3}}}
\newcommand\ebinddecl[3]{\econs{#1}{{#2}:{#3}}}
\newcommand\edecls[1]{\mathsf{decls}({#1})}
\newcommand\edefs[1]{\mathsf{defs}({#1})}

% Substitutions
\newcommand\snil\enil
\newcommand\sent[2]{[{#1}={#2}]}
\newcommand\scons[3]{{#1}\sent{#2}{#3}}
\newcommand\slook[2]{{#1}({#2})}

% Term operations
\newcommand\subst[2]{\{{#1}/{#2}\}}
\newcommand\repl[2]{\subst{#1}{{#2}}}
\newcommand\conv{\equiv}

% Languages
\newcommand\lang[1]{\textsf{#1}}
\newcommand\LF{\lang{LF}}
\newcommand\XLF{\lang{XLF}}
\newcommand\XLFa{\lang{XLF$_a$}}
\newcommand\NLF{\lang{NLF}}

\newcommand\XLFmod[1]{#1}
\newcommand\XLFamod[1]{\underline{#1}}

\global\def\thelangmod{UNDEFINED}
\def\inXLFa{\def\thelangmod{\XLFamod}}
\def\inXLF{\def\thelangmod{\XLFmod}}

% Metavariables (as appear in the rules & syntax description)

\newcommand\mk{\thelangmod{K}}
\newcommand\mmk{\thelangmod{K'}}
\newcommand\mmmk{\thelangmod{K''}}
\newcommand\mf{\thelangmod{A}}
\newcommand\mmf{\thelangmod{B}}
\newcommand\mmmf{\thelangmod{C}}
\newcommand\mmmmf{\thelangmod{D}}
\newcommand\mo{\thelangmod{t}}
\newcommand\mmo{\thelangmod{u}}
\newcommand\mmmo{\thelangmod{v}}
\newcommand\mh{\thelangmod{h}}
\newcommand\mmh{\thelangmod{h'}}
\newcommand\mmmh{\thelangmod{h''}}
\newcommand\ma{\thelangmod{l}}
\newcommand\mma{\thelangmod{m}}
\newcommand\mmma{\thelangmod{n}}
\newcommand\ms{\thelangmod{\sigma}}
\newcommand\mms{\thelangmod{\rho}}
\newcommand\mmms{\thelangmod{\theta}}
\newcommand\me{\thelangmod{\Gamma}}
\newcommand\mme{\thelangmod{\Delta}}
\newcommand\mmme{\thelangmod{\Xi}}
\newcommand\msi{\thelangmod{\Sigma}}

% Judgements
\newcommand\jlang[3]{{#2}\vdash_{\lang{#1}}{#3}}
\newcommand\jlangt[4]{{#2}\vdash_{\lang{#1}}{#3}:{#4}}
\newcommand\jlangA[3]{{#2}\vdash_{\lang{#1}}{#3}\mathsf{\ type}}
\newcommand\jlangK[3]{{#2}\vdash_{\lang{#1}}{#3}\mathsf{\ kind}}

\newcommand\jlft[3]{\jlangt{LF}{#1}{#2}{#3}}
\newcommand\jlfA[2]{\jlangA{LF}{#1}{#2}}
\newcommand\jlfK[2]{\jlangK{LF}{#1}{#2}}

\newcommand\jxlft[3]{\jlangt{XLF}{#1}{#2}{#3}}
\newcommand\jxlfA[2]{\jlangA{XLF}{#1}{#2}}
\newcommand\jxlfK[2]{\jlangK{XLF}{#1}{#2}}

\newcommand\jnlft{\jlangt{}}
\newcommand\jnlfA[2]{\jlangA{}{#1}{#2}}
\newcommand\jnlfK[1]{{#1}\ \mathsf{kind}} %on peut pas utiliser \jlangK

\newcommand\jnlfargs\jnlft
\newcommand\jnlfconvA[3]{\jnlfA{#1}{{#2}\conv{#3}}}
\newcommand\jnlfconvt[4]{\jnlft{#1}{{#2}\conv{#3}}{#4}}
\newcommand\jnlfconve[3]{\jlang{}{#1}{{#2}\conv{#3}}}

\newcommand\jindex[3]{{#2}\vdash_{#1}{#3}}

\newcommand\jannot[4]{\jindex{#1}{\inXLFa
    #2}{{\inXLF{#3}}\rightarrow{\inXLFa{#4}}}}
\newcommand\jannotK{\jannot}
\newcommand\jannotA{\jannot}
\newcommand\jannott{\jannot}
\newcommand\jannoth[5]{\jannot{#1}{#2}{#3}{{#4} : {#5}}}
\newcommand\jannotl[7]{\jannot{#1}{\pair{#2}{{#3}:{#4}}}{#5}{{#6}:{#7}}}
\newcommand\jannots{\jannot}

% Simple judgements with no environments (NLF)
\newcommand\jwf[1]{{#1}\ \mathsf{wf}}

% Functions
\newcommand\typeof[1]{{#1}\uparrow}
\newcommand\nlfgo[2]{{#1}|_{#2}}
\newcommand\nlflook\slook

\begin{document}

\conferenceinfo{ICFP '11}{September 19--21, Tokyo, Japan}
\copyrightyear{2011}
\copyrightdata{[to be supplied]}

%\titlebanner{banner above paper title}        % These are ignored unless
%\preprintfooter{short description of paper}   % 'preprint' option specified.

\title{Incremental type-checking}
%\subtitle{Subtitle Text, if any}

\authorinfo{Matthias Puech}
           {University of Bologna, University Paris Diderot}
           {puech@cs.unibo.it}
\authorinfo{Yann Régis-Gianas}
           {University Paris Diderot, CNRS, and INRIA}
           {yrg@pps.jussieu.fr}

\maketitle

\begin{abstract}

%% Four sentences:
%% State the problem
%% Say why it’s an interesting problem
%% Say what your solution achieves
%% Say what follows from your solution

\end{abstract}

\category{CR-number}{subcategory}{third-level}

\terms
term1, term2

\keywords
keyword1, keyword2

\section{Introduction}

%% Describe the problem (Use an example)
%% State your contributions (Bulleted list of contributions)

% \mrem{Une intro que j'avais écrit il y a longtemps et que je sais
%   pas où mettre. Sûrement à jeter\ldots}

% Programs are nowadays faster to write and safer to execute, thanks to
% high-level programming concepts and expressive type systems. Still,
% one aspect of programming didn't change very much---if not at
% all---since the early days of high-level programming: we still write
% programs in text files and when we are done, we let the compiler check
% and compile them.  Though, every programmer knows that writing a
% program longer than a few dozen of lines is a highly non-linear task
% involving constant experiments, fixes, rollback on previous
% modifications etc., all of those eventually validated by some a
% posteriori criterion: execution and ``empirical'' test, and
% increasingly complex type-checking and type inference. In a nutshell,
% programmers spend more time \emph{editing} than \emph{writing}. Yet,
% this aspect of their day-to-day workflow is not, or poorly taken into
% account in a developer's toolchain:
% \begin{itemize}
% \item To avoid recompiling a whole program upon a change, we split
%   them into files, and (re-)compile them separatly. Very often yet,
%   this separation serve different purpose: abstraction, splitting into
%   logical units, namespace management\ldots\ Why should these purposes
%   coincide?

% \yrem{On propose quelque chose d'encore plus extreme: faire descendre
% les VCS dans le type-checker. Voir deux paras plus loin.}

% \item To keep track of past versions of a program, we use text-based
%   version control systems that have no awareness of the actual content
%   of the file. In a sense, the non-linearity of the task is managed
%   outside the scope of the language, by manual, textual
%   transformations. Isn't there a lot to gain in making these tools
%   syntax-aware, even semantics-aware?

% \yrem{Je comprends mais je ne sais pas trop ou tu veux en venir.}

% \item As a result, all the semantic information that could be
%   available to the programmer and act as a \emph{guidance tool} to
%   write programs (e.g. type informations) are only used by the
%   type-checker as a \emph{repressive} measure against programming
%   errors (type errors).
% \end{itemize}

% These observations apply to the usual, wide-spread programming
% languages, but even if the current toolchain in use up to now was a
% reasonable approximation to the incremental interaction with the
% compiler intended by the user, they become \emph{a fortiori} necessary
% when it come to the new, demanding languages embedding formal
% verification aspects, like proof assistants or programming languages
% with rich type system. In these emerging languages, interaction with a
% type- or proof-checker is unavoidable due to the difficulty of writing
% correct proofs without guidance from the system, and the constant
% modification of the source makes necessary to observe incrementally
% the effect of a small change on the whole edifice. Moreover, as the
% complexity of the object to check increases, we cannot afford anymore
% to rely on the usual unit of compilation, the file, a need a
% finer-grained tool. 

% \yrem{On pourrait parler de l'utilisation de l'incrementalite pour le type-checking
% parallele et aussi pour le developpement concurrent.}

% Current text-based version control technology is designed to help
% concurrent and collaborative development but are not used inside
% compilers or proof checkers. There is a good reason for that: a
% representation of changes at the textual level is not structured
% enough to be used easily and efficiently by such symbolic tools. They
% need instead a more structured representation of changes with a clear
% meaning with respect to the language semantics. For instance, if a
% formal argument of a function has been renamed in a source file
% edition, a textual representation would refer to a change in every
% line that contains an occurrence of that formal argument. A proof
% kernel would prefer a higher level description of that edition from
% which it would be simple to deduce that no proof has to be rechecked
% because the old and the new version of the development are
% $\alpha$-equivalent.

% \yrem{Prochain para: LF. Finir sur les petits problemes de LF.}

% % Expliciter clairement les contributions du papier:
% % Une extension de LF avec des variables meta + lambda-selecteur-de-boite. (SLF)
% % Un langage pour representer efficacement un ensemble de termes validés. (NLF)
% % Une compilation de SLF vers NLF qui vérifie le bon typage.
% % Un typage incrémental par proof-search dans NLF.
% % La méta-théorie de NLF qui pose des bases pour la formalisation des VCS sémantiques.
% % Un prototype.

% We propose to devise here a data structure for repositories of
% programs and proofs allowing to take advantage of the incremental
% nature of a programmer's workflow.

\section{Incremental type-checking} % (1 page)

%% Here is a problem

\subsection{Why specialize incremental computation?}
%% It’s an unsolved problem

\subsection{Applications}
%% It’s an interesting problem

\section{Why a logical framework?} % (2 pages)
%% Here is my idea

\subsection{The essence of memoization}

\subsection{A metalanguage for sharing derivations}

% exemple en SLF

% Presentation de NLF.

\section{Repository data-structure and interaction}

%% My idea works (details, data)

\subsection{Architecture of the system}

\subsection{From \LF\ to \NLF}

\subsubsection{The source language}

\begin{figure}
  \inXLF
  \begin{align*}
    \mk &\gequal { \prd\mv\mf\mk \gor \type } \\
    \mf &\gequal { \prd\mv\mf\mmf \gor \lapp\mcf\ma } \\
    \mo &\gequal { \lam\mv\mf\mo \gor \lapp\mh\ma \gor \mmeta \gor \obox\mo\mv\ms } \\
    \mh &\gequal { \mo \gor \mco \gor \mv } \\
    \ma &\gequal { \lnil \gor \lcons\mo\ma } \\
    \ms &\gequal { \snil \gor \scons \sigma\mv\mo } \\[0.5em]
    \me &\gequal { \enil \gor \ebinddecl \Gamma\mv\mf } \\
    \msi &\gequal { \enil \gor \ebinddecl\msi\mco\mf \gor
      \ebinddecl\msi\mcf\mk }
  \end{align*}
  \caption{\XLF: Spine-form \LF\ plus the \textsf{Box} and \textsf{Meta} constructs}
\end{figure}

\subsubsection{Typing annotation and argument naming}

The first step in the translation is the embedding of \XLF\ into a
language with two small variations:
\begin{itemize}
\item application spines are annotated with their types, as well as
  metavariables and
\item each argument of such spines bears the name of the ``formal
  argument'' in the type of the head. For example, the function $t :
  \prd x A \prd y B C$ applied to terms $u$ and $v$ will be written
  $\laapp{t}{\lncons x u {\lnsing y v}} C$.
\end{itemize}

While the first point is easily justified by above arguments, the
second will be motivated in the next section. Informally, naming all
arguments with their expected variable in the type allow to forget
their order in a list $l$; a side-effect of this choice is discussed
below.

We thus modify the syntax of \XLF\ to obtain \XLFa (Fig. \ref{fig:XLFa}):

\begin{figure}
  \inXLFa
  \begin{align*}
    \mk &\gequal { \prd\mv\mf\mk
      \gor \type } \\
    \mf &\gequal { \prd\mv\mf\mmf
      \gor \memph{\laapp\mcf\ma\mk} } \\
    \mo &\gequal { \lam\mv\mf\mo
      \gor \memph{\laapp\mh\ma\mf}
      \gor \memph{\typ\mmeta\mf}
      \gor \obox\mo\mv\ms } \\
    \mh &\gequal { \mo
      \gor \mco
      \gor \mv } \\
    \ma &\gequal {\lnil
      \gor \memph{\lncons \mv\mo\ma}} \\
    \ma &\gequal { \lnil
      \gor \lcons\mo\ma } \\
    \ms &\gequal { \snil
      \gor \scons \sigma\mv\mo } \\[0.5em]
    \me &\gequal { \enil
      \gor \ebinddecl \Gamma\mv\mf } \\
    \msi &\gequal { \enil
      \gor \ebinddecl\msi\mco\mf
      \gor \ebinddecl\msi\mcf\mk }
  \end{align*}
  \caption{\XLFa: \XLF\ with type annotations and named arguments}
\label{fig:XLFa}
\end{figure}
Let us move on to the translation (Fig. \ref{fig:XLF_XLFa}). As usual,
it is parameterized implicitely by a signature $\Sigma$, unchanged
thus omitted in the rules. The judgements are first enumerated. All
the variables in them belong to the \XLFa\ world, except for the
\underline{underlined} one, which belongs to \XLF.

Since we are annotating terms with types, you will not be surprized
that the structure of the translation will closely follow the typing
rules of \XLF, except that the object () judgement ``return'' an \XLFa %TODO

\newcommand\repo{\mathcal R}
\begin{figure}
  \it
  \paragraph{Judgements}
  \begin{gather}
    \jannotK\repo\me\mk\mmk \\
    \jannotA\repo\me\mf\mmf \\
    \jannott\repo\me\mo\mmo \\
    \jannoth\repo\me\mh\mmh\mf \\
    \jannotl\repo\me\mma\mf\ma\mmma\mmf \\
    \jannots\repo\me\ms\mms
  \end{gather}
  Kind
  \begin{mathpar}
    %% kind
    \infer{ }{\jannotK\repo\me\type\type}
    \and
    \infer{
      \jannotK\repo\me\mf\mmf \and
      \jannotK\repo{\ebinddecl\me\mv\mf} \mk\mmk
    }{
      \jannotK\repo\me{\prd\mv\mf\mk}{\prd\mv\mmf\mmk}
    }
  \end{mathpar}
  Family
  \begin{mathpar}
    %% fam
    \infer{
      \jannotl\repo\me\lnil{\elook\msi\mcf} \ma\mma\mk
    }{
      \jannotK\repo\me{\lapp\mcf\ma}{\laapp\mcf\mma\mk}
    }
    \and
    \infer{
      \jannotA\repo\me\mf\mf \and
      \jannotA\repo{\ebinddecl\me\mv\mf}\mmf\mmf
    }{
      \jannotA\repo\me{\prd\mv\mf\mmf}{\prd\mv\mf\mmf}
    }
  \end{mathpar}
  Object
  \begin{mathpar}
    %% obj
    \infer{
      \jannotA\repo\me\mf\mf \and
      \jannott\repo{\ebinddecl\me\mv\mf}\mo\mo
    }{
      \jannott\repo\me{\lam\mv\mf\mo}{\lam\mv\mf\mo}
    }
    \and
    \infer{
      \jannoth\repo\me\mh\mh\mf \and
      \jannotl\repo\me\lnil\mf\ma\ma\mf
    }{
      \jannott\repo\me{\lapp\mh\ma}{\lapp\mh\ma}
    }
    \and
    \infer{
      \jannott{\nlfgo\repo\mv}\me\mo\mo \and
      \jannott\repo\me\ms\ms
    }{
      \jannott\repo\me{\obox\mo\mv\ms}{\obox\mo\mv\ms}
    }
    \and
    \infer{ }{
      \jannott\repo\me{\meta x}{\mameta {\nlflook\repo\mmeta}}
    }
  \end{mathpar}
  Arguments
  \begin{mathpar}
    %% args
    \infer{ }{
      \jannotl\repo\me\ma\mf\lnil\ma\mf
    }
    \and
    \infer{
      \jannott\repo\me\mo\mo \and
      \jannotl\repo\me{\lncons\mv\mo\ma}\mmf\ma\mma\mmmf
    }{
      \jannotl\repo\me\ma{\prd\mv\mf\mmf}{\lcons\mo\ma}\mma\mmmf
    }
  \end{mathpar}
  Head
  \begin{mathpar}
    %% head
    \infer{ }{
      \jannoth\repo\me\mo\mo{\elook\me\mo}
    }
    \and
    \infer{ }{
      \jannoth\repo\me\mcf\mcf{\elook\msi\mcf}
    }
    \and
    \infer{ }{
      \jannoth\repo\me\mco\mco{\elook\msi\mco}
    }
    \and
    \infer{
      \jannott\repo\me\mo\mo
    }{
      \jannoth\repo\me\mo\mo{(\typeof\mo)}
    }
  \end{mathpar}
  Substitution
  \begin{mathpar}
    %% subst
    \infer{ }{
      \jannots\repo\me\snil\snil
    }
    \and
    \infer{
      \jannott\repo\me\mo\mo \and
      \jannots\repo\me\ms\ms
    }{
      \jannots\repo\me{\scons\ms\mv\mo}{\scons\ms\mv\mo}
    }
  \end{mathpar}
  \paragraph{where}
  \inXLFa
  \begin{align*}
    \typeof{(\lam\mv\mf\mo)} &= \prd\mv\mf{(\typeof\mo)} \\
    \typeof{(\laapp\mh\ma\mf)} &= \mf
  \end{align*}
  \caption{From \XLF\ to \XLFa: type-annotations on spines}
  \label{fig:XLF_XLFa}
\end{figure}

\subsubsection{Non-positional calculus}

\subsection{Interaction with a repository}

% Metathéorie de NLF

% Applications (typage incrémental, quelques premières pistes pour commit/merge)

\section{Related works} % (1-2 pages)

%% Here’s how my idea compares to other people’s approaches


% Related works (1-2 pages) (à ne surtout pas négliger!)

\section{Conclusion} % (0.5 page)

% Conclusion and further work (0.5 page)
% Dans le future work, en vrac:
%    - un meta-langage au dessus de NLF = un langage de tactique typé pour Coq?
%    - des algorithmes de parsing incrémentaux
%    - modélisation des conflits et de leur résolution dirigée par des types.
%    - intégration dans Coq pour la parallélisation du noyau. 
%   -  certification du checker pour être intégré dans le noyau de Coq 
\bibliographystyle{abbrvnat}

\end{document}
