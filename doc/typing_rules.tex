\documentclass[12pt]{article}
\usepackage{mathpartir}
\usepackage{ucs}
\usepackage[utf8x]{inputenc}
\usepackage{mathpartir}
\usepackage{color}
%\usepackage[pdftex,backref=page,colorlinks=true]{hyperref}
\usepackage{amsmath, amstext, amsthm, amsfonts}
\usepackage{stmaryrd}
\usepackage[protrusion=true,expansion=true]{microtype}
\usepackage{fullpage}

\title{A kernel for incremental type-checking}

\newcommand{\sort}{\textsf{s}}
\newcommand{\gor}{\ |\ }
\newcommand{\gdecl}[2]{{#1}\ &::=\ {#2}}
\newcommand{\subst}[2]{\{#1/#2\}}
\begin{document}

\maketitle

\paragraph{Introduction}

Representing syntax and logics is nicely done in a \emph{logical
  framework} like LF: both the syntactic elements and the typing
derivations sit in the same tree structure, and both can be rechecked
at the same time. The PTSes are a concise presentation of a family of
logical frameworks.

For the purpose of incremental type checking though, our needs are a
little bit different: first, we need to record, that is to name all
intermediate values of our developments, so as to be able to address
and reuse them multiple times. Secondly, we need to make sure that
intermediate values are not recorded twice, which would imply that we
will verify the same subterm twice (precisely what we are trying to
avoid). These two conditions together entail that we are looking to
represent syntactic and typing objects as a \emph{DAG} rather than a
tree, and moreover a DAG with \emph{maximal sharing} among its
sub-DAGs.

Let's take the PTS approach, and tailor it to our needs. We make
several changes to the usual presentation:
\begin{itemize}
\item In a first iteration of our project, we do not need computations
  to take place within our DAGs. No $\lambda$'s, no conversion
  rule. We stratify the syntax of the system into terms and types.
\item Compound terms are what we want to avoid. Every term should be a
  flat application of variables. 
\item Finally we need a way to record these intermediate definitions:
  we introduce a new kind of product, singleton products, which
  contain the definition of the new object along with its name and
  type. Environments are extended for these definitions.
\end{itemize}

\paragraph{Parameters}

As for a regular PTS, our system depends on three variables:

\begin{itemize}
  \item$\mathcal S$ a set of sorts \sort,
  \item$\mathcal A \in \mathcal S^2$ the axioms of the system,
  \item$\mathcal R \in \mathcal S^3$ the allowed products.
\end{itemize}

\paragraph{Syntax} There are three syntactic categories: applicative
terms $a$, types $t$ and environments $\Gamma$. Environments look like
list, but in fact they aren't (see below).

\begin{align*}
 \gdecl{a}{x \gor a\ x } \\
 \gdecl{t}{a \gor \sort \gor (x:t)\cdot t \gor (x=a:t)\cdot t} \\
 \gdecl{\Gamma}{\cdot \gor \Gamma[x:t] \gor \Gamma[x=a:t]}
\end{align*}

Following the standard syntax, we denote by $(x:t)\cdot u$ the
dependent product (elsewhere written with a $\Pi$). The new construct
$(x=a:t)\cdot u$ is a kind of singleton type, whose value is $a$. Such
a singleton product should be read as ``the (implicit) function taking
as argument any values syntactically equal to $a$ of type $t$
(\emph{i.e.} a pedantic way to say $a$). 

We write as usual $(x_1:t_1)(x_2:t_2)\ldots(x_n:t_n)\cdot t$ for
$(x_1:t_1)\cdot(x_2:t_2)\cdot\ldots\cdot(x_n:t_n)\cdot t$, $t \to
u$ for $(x:t)\cdot u$ where $x\notin u$, and $(x\ y : t)\cdot u$ for
$(x:t)(y:t)\cdot u$. The lookup in the environment
is written either $\Gamma(x):t$, denoting ``$\Gamma$ contains a
declaration $[x:t]$ \emph{or} a definition $[x=a:t]$'', or
$\Gamma(x)=a:t$ denoting ``$\Gamma$ contains a definition
$[x=a:t]$''. Capture-avoiding renaming of variables in a term $t$ is
written $t\subst{x}{y}$ for ``replace all occurences of $x$ by $y$''
and is defined the usual way.

We maintain the strong invariant that the environment does not contain
two variables of the same name, nor two syntactically equal
definitions (the $a$ part is unique). More on that in the
implementation part.

\paragraph{Typing} The typing of our system is a restriction of the
PTS's typing rules with the additional two rules for typing a singleton
product and the application of a singleton argument (featuring the
syntactic verification in the environment). It relies on two distincts
jugements:
\begin{itemize}
\item $\Gamma\vdash a : t$ ($a$ has type $t$ in $\Gamma$),
\item $\Gamma\vdash t : \sort$ ($t$ has sort $s$ in $\Gamma$).
\end{itemize}
The rules are:
\begin{mathpar}
  \infer[Axiom]{
  }{\Gamma\vdash\sort_1:\sort_2}
  \quad(\sort_1,\sort_2)\in\mathcal A
  \and
  \infer[Init]{
    \Gamma(x):t
  }{\Gamma\vdash x:t}
  \and
  \infer[Prod]{\Gamma\vdash t:\sort_1 \and 
    \Gamma[x:t]\vdash u:\sort_2
  }{\Gamma\vdash(x:t)\cdot u : \sort_3}
  \quad(\sort_1,\sort_2,\sort_3)\in\mathcal R
  \and
  \infer[Sing]{\Gamma\vdash a:t \and
    \Gamma\vdash t:\sort_1 \and
    \Gamma[x=a:t]\vdash u:\sort_2
  }{\Gamma\vdash(x=a:t)\cdot u : \sort_3} 
  \quad(\sort_1,\sort_2,\sort_3)\in\mathcal R
  \and
  \infer[App]{
    \Gamma\vdash a:(y:t)\cdot u \and \Gamma(x):t
    }{\Gamma\vdash a\ x : u\subst{x}{y}}
  \and
  \infer[Skip]{
    \Gamma\vdash a:(y=b:t)\cdot u \and \Gamma(x)=b:t
    }{\Gamma\vdash a : u\subst{x}{y}}
\end{mathpar}

A term $t$ represents a state of the repository at a given moment. It
contains the declaration of the object language's syntax, its typing
rules, the syntactic objects and derivations recorded in it along with
the patches that let the repository evolve. If $\cdot\vdash t:\sort$,
then the repository is well-typed.

\paragraph{Implementation}

The main problem when it comes with implementation is to guarantee the
unicity of declarations and definitions in the environment. One
approach would be to use the locally-nameless method, along with a
special type of free names: A term is stored in deBruijn syntax, but
as soon as a variable is introduced, it takes a real name. So in fact,
environments are not syntactic list objects, instead the operations
$\Gamma[x:t]$ and $\Gamma[x=a:t]$ are the partial meta-operations of
adding new entries into the environment, which returns a new, unique
name for them. The last meta step is to substitute $x$ for this new
name.

How to choose the names in order for each of them to be uniquely bound
to their content (for definitions)? Let's follow \textsf{git} on this:
each definition is given for name the hash of its content
(definiens). In turn, the content of a definition is itself an
application of variables, which have been given unique names earlier,
so taking the hash of the content guarantees that the whole sub-DAG is
contained in the name.

The receipt to choose a name at introduction-time (\textsc{Sing} or
\textsc{Prod}) is:
\begin{itemize}
\item If we introduce a \emph{declaration}, it has no content. Let's
  assign it a big random hash for a name: it will ensure its
  uniqueness.
\item If we introduce a \emph{definition}, then we compute the hash of
  its content. If this hash is already present in the environment, we
  don't introduce anything and take this hash as a name. Otherwise we
  introduce the definition and assign it the hash as a name.
\end{itemize}

\paragraph{An example}

Let's put aside the problem of configuring the PTS for the moment,
since it's not clear yet: we work in system U \emph{i.e.}
$\langle\{*\},\{*:*\},\{(*,*,*)\}\rangle$.

We are going to construct a repository gradually as a big term, but
stopping at some steps to explain the construct. The beginning of the
term will consist in the declaration of the language (a small
arithmetic language). Keep in mind that this simply typed language is
just an example: more evolved languages can and will be declared, even
on-the-fly as part of a commit!
\newcommand\nat{\mathbb N}
\newcommand\prop{prop}
\begin{align}&
  (\prop : *)(\nat : *)(0:\nat)(S:\nat) \\&
  (\land\ \lor\Rightarrow\ : \prop \to \prop \to \prop) \\&
  (+\ -\ : \nat\to\nat\to\nat)
  (=\ \neq\ \leq:\nat\to\nat\to \prop )
  \intertext{Now let's begin the definition of the expression
    $2+2=4$. The syntax restricts us to define it step-by-step:} &
  (1 = S\ 0 : \nat)(2 = S\ 1 : \nat)(3 = S\ 2 :
  \nat)(4=S\ 3 : \nat) \\&
  (t = (+)\ 2\ 2 : \nat) (p : (=)\ t\ 4) \\&
  \intertext{Forever, the name $p$ will be associated to that
    expression. Let's mark it with a name. First we introduce -- on
    the fly -- the marking feature to our repository (how cool is that?):} &
  (commit : *)
  (mark : \prop\to string\to commit)
  \intertext{The type $string$ is predefined in some way, and the
    choice of $prop$ as the argument to $mark$ fixes that we want to
    record propositions as commits, not naturals: it is in some sense
    the ``tip'' of our syntax. Now we can mark, or commit that $p$: } &
  (x = mark\ p\ \mathtt{``Cool\ equation}" : commit)
  \intertext{Now let's make it evolve to $2+2=1+3$, and commit the result} &
  (u = (+)\ 1\ 3 : \nat)(q = (=)\ t\ u : \prop) \\&
  (y = mark\ q\ \mathtt{``Also\ true}")
  \intertext{We could stop here: we have a nice object for commits,
    and a way to verify by typing that our repository is valid. Let's
    try to exploit the type system a little bit more: We  now
    introduce a higher-order object -- a
    patch maybe? -- taking a piece of syntax matching a certain form
    and constructing another object. In that case, it takes an
    equation and transforms it into an inequation:} &
  (P : (t\ u:\nat)(p=(=)\ t\ u:\prop)(q=(\leq)\ t\ u)\cdot mark\ q\
  \mathtt{``A\ fortiori}")
  \intertext{Now we can show that this patch applies to our
    repository by filling in the missing items:} &
  (z = P\ t\ u : (p=(=)\ t\ u:\prop)(q=(\leq)\ t\ u)\cdot mark\ q\
  \mathtt{``A\ fortiori}")
\end{align}
When the type of $P\ t\ u$ will be typed, the variable $p$ will be
given the same name as the previously-introduced $q$, because their
content match; whereas the new $q$ will be introduced into the
environment, and given a new unique name.
\\

The status of commits, patches etc. is still not very clear and the
previous example should be taken with a grain of salt, but this
hopefully gives a hint of what is possible with a dependently-typed,
memoizing language.

\end{document}
