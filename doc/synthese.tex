\documentclass{article}
\usepackage[francais]{babel}
\usepackage[utf8]{inputenc}

\begin{document}

\title{Synthèse des idées sur \textsc{Gasp}}

Ce document synthétise les idées sur le développement du
\textit{framework} de patchs sémantiques \textsc{Gasp}, dans le but de
décider efficacement des directions de recherche à explorer.

\section{Le noyau}

\section{Les outils}

\subsection{Édition avec historique}

\subsubsection{Édition \textit{via} des vues}

La syntaxe concrète au choix. Un choix intéressant~: les constructeurs des termes du langage,
des noms pour des transformations locales (de refactoring), des annotations d'aide à la 
reconstruction de l'objet interne (généralisation des annotations de type dans le cas
de l'inférence de type). 

\subsubsection{Algorithme de \textit{diff}}

Un outil pour l'incrémentalité. Une application du cadre de la programmation bidirectionnelle (?). 

\subsection{Développement collaboratif}

Qu'est-ce qu'une branche? Qu'un merge? Toutes les opérations de \textsc{Git}, de \textsc{Darcs}? 
Une hypothèse~: la métathéorie du langage de patchs
fournit des outils méta semblant capturer ces notions. Une auto-application du système
fait apparaître un langage de description de l'évolution d'un ensemble d'historique. 

\subsection{Unification du système de modules (de seconde classe) et du système de construction}

\section{Certification en \textsc{Coq}}

Coq pour donner une interprétation au langage et pour construire un noyau de primitives
certifiées sur lesquelles on peut s'appuyer pour construire des transformations plus 
complexes. 

\section{Plan d'attaque}

\end{document}
