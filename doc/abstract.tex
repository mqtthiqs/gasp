\documentclass{article}

\usepackage{ucs}
\usepackage[utf8x]{inputenc}
\usepackage{mathpartir}
\usepackage{color}
\usepackage[pdftex,backref=page,colorlinks=true]{hyperref}
\usepackage{amsmath, amstext, amsthm, amsfonts}
\usepackage{stmaryrd}
\usepackage{geometry}
\usepackage{natbib}
\usepackage{fancyhdr}
\input xy
\xyoption{all}
\usepackage[protrusion=true,expansion=true]{microtype}

\definecolor{bwgreen}{rgb}{0.183,1,0.5}
\definecolor{bwmagenta}{rgb}{1,0.169,0.909}
\definecolor{bwblue}{rgb}{0.317,0.161,1}
\hypersetup{
  linkcolor=blue,
  citecolor=blue
}

\newcommand{\sort}{\textsf{s}}
\newcommand{\gor}{\ |\ }
\newcommand{\gdecl}[2]{{#1}\ &::=\ {#2}}
\newcommand{\subst}[2]{\{#1/#2\}}

\title{Semantic Patches
  % pas sûr que le nom soit encore adapté
}
\author{Matthias Puech $\quad\Leftrightarrow\quad$ Yann Régis-Gianas}
\date{}

\begin{document}
\maketitle

\section{Motivations}

It is interesting to investigate the possible relationship between the
very \emph{methodologies} or even \emph{daily workflow} employed in
both fields of mathematics and computer science: one trying to prove a
theorem, the other constructing a program to fulfill a task. How does
one or the other elaborates his object of study? What kind of \emph{a
  posteriori} modification is he prone to doing? What do these
modifications imply on the validity of the whole edifice or the other
way around, how do one rely on existing work to build up new results?
Finally, how to relate how both scientists collaborate in a team?

Recent efforts in the formalization of mathematical results have
naturally led to these questions and many of them remain largely
unanswered, but the tendency seems to be to adapt existing methods
coming from software development, witness for example the recent
introduction of modules, file-based scripts and separate compilation
in proof assistants like \textsf{Coq} or \textsf{Matita}, the use of
dependency management tools (\textsf{make}) or version control system
(\textsf{git}, \textsf{svn}) to build and manage versions of a
project. Both for the development of proofs or programs, these tools
attempt to cope with the fact that most of a mathematician or
programmer's time is actually spent \emph{editing}, not
\emph{writing}.

As intelligent and widely adopted as this tool chain has gotten, we
believe that they are not adapted for the new, demanding requirements
of proof developments. Indeed, whereas compilation of a program is
usually fast enough for the programmer to rely on the usual
interaction loop ((edit; compile)*; commit)*, the operation of proof
checking is usually too expensive computationally to mimic this
workflow. But even besides the time factor, this ``traditional'' way
of formalizing mathematics hinders the process of mathematical
discovery process: once a concept contained in a file is compiled, it
is considered frozen and any changes to it require the recompilation
of the whole file; the linearity of the development also gives no room
for alternate, inequivalent definitions. This fact has nonetheless
been shown to be of crucial to the mathematical discovery process by
\cite{lakatos1964proofs}, and we believe that they should be taken
into account in the realization of mathematical assistants like
\textsf{Coq} or \textsf{Matita}.

We propose to discuss a small part of these questions, namely the
enhancement and adaptation of version control paradigms to the
management of mathematical repositories, to witness with more
precision the \emph{impact of changes}. By studying the dependencies
between objects in a proof, we can get a much finely grained vision of
what a future evolution would impact. In proof assistants based on the
propositions-as-types paradigm, Type Theory offers a powerful tool to
witness the effect of changes finely, \emph{i.e.} types. Eventually,
we hope to be capable of expressing evolutions on formal proofs, check
them efficiently and incrementally, and manage distributed
repositories of mathematics with it, guaranteeing \emph{by typing} the
stepwise global consistency of the repository.

In the light of the propositions-as-types paradigms where
proof-checking boils down to type-checking, we abstract from whether
we are talking about proofs or programs, so that our development could
eventually also be used in traditional software development.

\section{Methodologies}

We propose here to devise a system for \emph{semantic patches}. It
% TODO changer le nom?
substitutes the idea of textual transformation by:
\begin{itemize}
\item First preferring an abstract syntax tree representation instead
  of plain text;
\item Secondly embedding semantic, or typing data into the
  transformations, in order to be able to reason on them.
\item Finally, perform type-checking in an incremental way, that is
  type only the syntactic difference and reuse derivation for the
  already known subprograms.
\end{itemize}

We want to design a \emph{language of patches}, able to represent
% TODO changer le nom?
these transformations and their semantic properties, thus capturing
local syntactic changes in programs (the abstract syntax tree), as
well as their global effect on the whole project (the typing
derivations). We can see this goal as a refinement of the former idea
of ``dependency''.

\subsection{Data \emph{vs.} patch-oriented}

\subsection{Related Works}

\section{A kernel for incremental type-checking}

Representing syntax and logics is nicely done in a \emph{logical
  framework} like LF (\cite{harper1993framework}): both the syntactic
elements and the typing derivations can sit in the same tree
structure, and both can be rechecked at the same time, thanks to
dependent types.

For the purpose of incremental type checking though, our needs are a
little bit different: first, we need to record, that is to name all
intermediate values of our developments, so as to be able to address
and reuse them multiple times. Secondly, we need to make sure that
those intermediate values (sub-terms) are not recorded twice, so as to
not type-check them twice. Thus we are looking to represent syntactic
and typing objects as a \emph{DAG} rather than a tree, and ensure
\emph{maximal sharing} among its sub-DAGs.

Our system will have these properties \emph{wrt.} LF:
\begin{itemize}
\item In this first iteration of the project, we do not need
  computations to take place within our DAGs. Our syntax will then be
  restricted to product types and applications;
\item Every term should be a flat application of variables, so that we
  don't introduce compound terms without naming them;
\item Finally we need a way to record intermediate definitions: we
  introduce a new kind of binder, the equality binder. We maintain the
  invariant that not two same definition can sit in the context while
  typing.
\end{itemize}

\paragraph{Syntax} There are three syntactic categories: applicative
terms $a$, types $t$ and environments $\Gamma$.

\begin{align*}
 \gdecl{a}{x \gor a\ x } \\
 \gdecl{t}{a \gor \sort \gor (x:t)\cdot t \gor (x=a:t)\cdot t} \\
 \gdecl{\Gamma}{\cdot \gor \Gamma[x:t] \gor \Gamma[x=a:t]}
\end{align*}

Following the standard syntax, we denote by $(x:t)\cdot u$ the
dependent product (elsewhere written with a $\Pi$). The new construct
$(x=a:t)\cdot u$ is a definition, whose value is $a$. We write as
usual $(x_1:t_1)(x_2:t_2)\ldots(x_n:t_n)\cdot t$ for
$(x_1:t_1)\cdot(x_2:t_2)\cdot\ldots\cdot(x_n:t_n)\cdot t$, $t \to u$
for $(x:t)\cdot u$ where $x\notin u$, and $(x\ y : t)\cdot u$ for
$(x:t)(y:t)\cdot u$. The lookup in the environment is written either
$\Gamma(x):t$, denoting ``$\Gamma$ contains a declaration $[x:t]$
\emph{or} a definition $[x=a:t]$'', or $\Gamma(x)=a:t$ denoting
``$\Gamma$ contains a definition $[x=a:t]$''. Capture-avoiding
renaming of variables in a term $t$ is written $t\subst{x}{y}$ for
``replace all occurences of $x$ by $y$'' and is defined the usual way.

\paragraph{Typing} The typing of our system is a restriction of the
usual typing rules for dependently-typed systems with the additional
two rules for typing a singleton product and the application of a
singleton argument (featuring the syntactic verification in the
environment). It depends on: 
\begin{itemize}
  \item$\mathcal S$ a set of sorts \sort,
  \item$\mathcal A \in \mathcal S^2$ the axioms of the system,
  \item$\mathcal R \in \mathcal S^3$ the allowed products.
\end{itemize}
It relies on two distincts judgments:
\begin{itemize}
\item $\Gamma\vdash a : t$ ($a$ has type $t$ in $\Gamma$),
\item $\Gamma\vdash t : \sort$ ($t$ has sort $s$ in $\Gamma$).
\end{itemize}
The rules are:
\begin{mathpar}
  \infer[Axiom]{
  }{\Gamma\vdash\sort_1:\sort_2}
  \quad(\sort_1,\sort_2)\in\mathcal A
  \and
  \infer[Init]{
    \Gamma(x):t
  }{\Gamma\vdash x:t}
  \and
  \infer[Prod]{\Gamma\vdash t:\sort_1 \and 
    \Gamma[x:t]\vdash u:\sort_2
  }{\Gamma\vdash(x:t)\cdot u : \sort_3}
  \quad(\sort_1,\sort_2,\sort_3)\in\mathcal R
  \and
  \infer[Sing]{\Gamma\vdash a:t \and
    \Gamma\vdash t:\sort_1 \and
    \Gamma[x=a:t]\vdash u:\sort_2
  }{\Gamma\vdash(x=a:t)\cdot u : \sort_3} 
  \quad(\sort_1,\sort_2,\sort_3)\in\mathcal R
  \and
  \infer[App]{
    \Gamma\vdash a:(y:t)\cdot u \and \Gamma(x):t
    }{\Gamma\vdash a\ x : u\subst{x}{y}}
  \and
  \infer[Skip]{
    \Gamma\vdash a:(y=b:t)\cdot u \and \Gamma(x)=b:t
    }{\Gamma\vdash a : u\subst{x}{y}}
\end{mathpar}

A term $t$ represents a state of the repository at a given moment. It
contains the declaration of the object language's syntax, its typing
rules, the syntactic objects and derivations recorded in it along with
the patches that let the repository evolve. If $\cdot\vdash t:\sort$,
then the repository is well-typed.


\bibliographystyle{plainnat}

\bibliography{english}

\end{document}
