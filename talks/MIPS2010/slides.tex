% \documentclass[ignorenonframetext,red]{beamer}

\documentclass{article} \usepackage{beamerarticle}

\usepackage{ucs}
\usepackage{mathpartir}
\usepackage{amsfonts,amsmath,amscd}
\usepackage{stmaryrd}
\usepackage[utf8]{inputenc}
\usepackage[protrusion=true,expansion=true]{microtype}

\title{Towards typed repositories of proofs}

\date{July 10, 2010}

\author[Matthias Puech \& Yann Régis-Gianas] {
Matthias Puech\inst{1,2} \and Yann Régis-Gianas\inst{2} \\
{\small \url{puech@cs.unibo.it}} \and {\small \url{yrg@pps.jussieu.fr}}
}
\institute {
  \inst 1 {\small Dept. of Computer Science, University of Bologna} \and
  \inst 2 {\small University Paris 7, CNRS, and INRIA, PPS, team ${\pi}r^2$}
}

\setbeamertemplate{footline}[frame number]
\setbeamertemplate{navigation symbols}{}

\AtBeginSection[]
{\begin{frame}<beamer>{Outline}
    \tableofcontents[currentsection]
  \end{frame}
}

\usefonttheme{serif}

\begin{document}

\begin{frame}
  \titlepage
  \mode<article>{\newcommand\url\texttt\maketitle}
\end{frame}
\section*{Introduction}

This talk will be about some remarks and directions on the way the
edition of formal proof is done in proof assistants, especially those
based on type theory like \textsf{Coq} or \textsf{Matita}.

\begin{frame}{A paradoxical situation}
  \begin{itemize}
  \item Most time spent \emph{editing}, not \emph{writing}
  \item Workflow of formal mathematics largely inspired by software development
  \end{itemize}  
\end{frame}

Let me start with two simple observations relating the activity of
both the working mathematician and the programmer. A first, obvious
fact is that both spend more time \emph{editing} than \emph{writing}:
their workflow is highly non-linear, involving experiments, fixes,
backtrack on previous modifications\ldots, eventually validated or
invalidated by some criteria like the absence of bugs, the conformance
to a specifications on one side, or the validity of a proof, the
adequacy of a definition on the other.

The second observation is that since the advent of modern proof
assistants, large bodies of mathematics have been formalized and the
way people managed these developments is very similar to the one used
in software development: we write proofs in text buffers, split the
development into separate files

\begin{frame}{Outline}
  \tableofcontents
\end{frame}

\section{}

\subsection{Practical applications}

\section{A core meta-language for proof repositories}

\end{document}