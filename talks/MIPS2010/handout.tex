\documentclass[12pt]{article}
\usepackage{fullpage}
\usepackage{ucs}
\usepackage{mathpartir}
\usepackage{amsfonts,amsmath,amscd}
\usepackage{stmaryrd}
\usepackage[utf8x]{inputenc}
\usepackage[protrusion=true,expansion=true]{microtype}

\title{Towards typed repositories of proofs \\[0.6em] 
  \small \textsf{MIPS 2010}}

\date{July 10, 2010}

\author{Matthias Puech \and Yann Régis-Gianas}

\newcommand{\slide}{\vspace{1em} \hrule \vspace{1em}}

\begin{document}

\maketitle

Let me start with a simple observation on the activity of both the
working mathematician and the computer scientist. If we observe the
day-to-day work of these scientists, their workflow resembles a lot
that of researchers in experimental sciences: it is highly non-linear,
involving multiple parallel experiments, fixes, backtrack on previous
modifications\ldots\ eventually validated or invalidated by some
\emph{a posteriori} criterion, the absence of bugs, the validity of a
proof.

In other words, the mathematician and the programmer spend more time
\emph{editing} their developments non-linearly than monotonously
\emph{writing} new material.

I am going to present a work-in-progress, realized together with Yann
Régis-Gianas. It is far from a finished product, so feel free to
interrupt, comment or criticize. It will be about some observations
and directions to improve the way we edit and store formal proofs in
proof assistants.

\slide

We now have very capable tools available to reason about formal
languages and especially proof languages. Some of these tools are even
dedicated to the mechanization of the metatheory of proof languages.

But still, the way people managed large development of formal proofs
is very similar to the one used in software development for years: we
write proofs in text files, which are linear, and launch
proof-checking when we are done. To avoid re-checking the whole
development, we split it into files. To keep track of old versions, we
use line-based version control systems that have no awareness of the
actual content of the files. 

In a sense, the non-linearity of proof development is managed outside
the scope of the tool, by manual, textual transformations that are
often very unsatisfactory.

Couldn't we adapt the methods developed to reason about formal
languages to the language of the proof assistant itself? Could we
replace this legacy toolchain and make it language-aware, even
semantic-aware?

\slide

More precisely, here are some problems that arise with the traditional
tools when editing proofs. Whereas the usual edit, compile, commit
loop may be a satisfactory approximation when developing software, the
extra computational cost of proof-checking makes it unsatisfactory as
the development grows: often the time of recompilation is prohibitive.

But it is not only a matter of time: once a concept is proof-checked,
it is frozen and any future modification of it will require to
re-check a large part of the development, often much more than what is
actually required. We believe that this fact inhibits the use of proof
assistants as tools to help the mathematician in the discovery
process, as an experimentation tool, and not only to formally
transcribe paper proofs.

The same way, the linearity of proof scripts does not allow any
flexibility in the alternate definition of concepts.

On the other side, the way we store mathematical developments, that is
as textual files or textual diffs does not reflect neither their
syntax nor their semantics, and as I will try to show, there is a lot
to gain in changing this base representation.

In fact, some will say that this model of interaction wasn't even
adapted to programming in the first place.

\slide

We can then see a number of extra-logical features of modern proof
assistants as ways to cope with these problems. 

Instead of writing our complete development in one unique file, we can
split it in several files or modules and establish dependencies among
them: only those who changed (and their dependencies) have to be
rechecked upon a change.

But this separation serves two different purpose, and the fact that
they coincide is pure contingency: the purpose of splitting a
development into logical units, and the purpose of accelerating
recompilation upon changes.

Secondly, the traditional interaction loop facilitates local changes
by providing a global undo feature of the whole state of the tool. But
if I want to change this proof, I have to backtrack all the way here
(because the green region is locked), do my changes and then replay
the whole proofs in between, even if my change had no impact on this
one, even if I just added a comment.

So both these facilities provide a coarse approximation of the
dependency between objects. But this notion of dependency \emph{is} a
metatheoretical property of a logic, and it could be used to provide
an \emph{exact} way of dealing with the impact of changes, in
particular to guide proof-checking to recheck only minimal parts of
the developments.

This is what we will try to do in the following.

\slide

Here is a schematic view of the traditional way to manage formal
developments. The user writes scripts which are version controlled,
and these scripts are parsed and checked on demand. By pushing version
management inwards, we abstract from syntax, and more informations
become available. For instance we can compute the acyclic graph of
dependencies. At this level already, the situation is greatly
enhanced: syntax, and file-based development, get the place they
deserve: they are relegated to a tool for user interaction, the actual
development being stored and manipulated in an abstract form. We could
even imagine syntax being dynamically switchable, part of user
preferences!

But let's go one step further and embed version management into the
type-checker. We gain access to type informations at the level of
changes, that is, we can track changes, dependencies and types at the
same time, and decide to type-check only local changes of the
development and their global effect, in a minimal way. We can perform
\emph{incremental type-checking}.

\slide

In the following, I will present a preliminary implementation of these
ideas. It is a meta-language for incremental proof-checking, that is a
language in which one can define abstract syntaxes, logical rules and
proofs, and allows to check validity, express the incrementality of
the constructions, and their dependency. It can be viewed as a kernel
for a generic, typed version control system.

\slide

As an inspiration, let's see a very simple, common technique for
storing history of directories. We won't keep track of changes
\emph{per se}, but of similarities between version. Given a base
directory structure, already committed in its database, and a new
version of it with, let's say, one file changed -- \texttt{f3.txt}, we
stores the new version by linking all common subdirectory structure to
the previous version. This way, similarities are physically relalized
by this kind of ``compression''. This is possible if we give to every
node in the tree, directories and files, a unique name. So actually,
our database looks like this: it associates to every name, a content
(for files) or a list of names (for directories). The invariant is
that this database forms a DAG. This naming policy makes all content
adressable.

Moreover, similarities are easily tracked if every name is carefully
chosen to reflect the \emph{content} of a file or directory.  In
practice, a hash function is used to produce all names in the
database: files are given the hash of their content, and directories
are given the hash of the hash of all the files it contains,
recursively.

This way, we ensure that no two subdirectories with the same content
appear in the database, that is, we enforce maximal sharing among
subdirectories. In functional programming, this is called
hash-consing.

This is exactly the storage model of \textsf{git}, a popular version
control system.

Let's apply the same discipline to proofs, and validate them by
typing. What would we get?

\slide

On the same model, let's start with a proof of $C$ already committed,
and starting with this structure. $\pi_1$, $\pi_2$ and $\pi_3$ are the
rest of the proof, upwards. Each node is labelled with the logical
rule (or constructor) used, but also annotated by the judgement it
proves. Now let's imagine that we come up with a new, better proof for
$B$. Integrating it in the repository means, like before, to share all
common sub-derivations. But now, we don't need to retype the whole new
proof, just the new parts (in red). Like before, this is possible if
we give a name to all sub-derivations, and the database looks like
this: it associates with every name, how it is formed.

\slide

Let's construct a calculus for this. We start from the usual Pure Type
Systems, with abstraction, products, variables, applications and just
one sort for the sake of simplicity. Typing is done in the usual
name-type environment, by the usual judgement: ``In $\Gamma$, $t$ has
type $u$''. We modify it gradually, keeping its expressivity but
tailoring the syntax to name all subterms.

First we don't want to allow compound, deep applications. We remove
general application and make it a list of variables. Then, we need a
new binder to remember these flat applications and give a name to
them. It will behave like a \textsf{let} binding, and we add a
corresponding entry to the environment. Note that in terms, the type
is not present whereas it is in the environment.

To understand how incrementality is handled, we focus for the moment
on a minimal system. It happens that it is already interesting enough
without abstraction. For the moment, we remove it, and with it all
notion of computation.

These simple modifications allow us to see this system as a typed
repository of proofs with a last modification: the usual judgement
becomes ``From repository $\Gamma$, term $t$ of type $u$ leads to
repository $\Delta$''. The analogy here is: $\Gamma$ is the current
repository, $t$ is a piece of proof that has to be checked and
committed, it can have redundancies (it's not maximally shared), and
integration of $t$ into $\Gamma$ leads to $\Delta$, which is maximally
shared.

We thread the environment along the typing process and return it as
the result.

\slide

I will present briefly some of the typing rules of this calculus. Here
is for example the usual product: we thread the return environment in
its main branch. Here is the usual init rule, we return the enlarged
environment.

\slide

The rules for the equality binder are two. The first resembles a lot
that of the \textsf{let} binding: we check the body, then the rest by
putting a new definition in the environment. But it applies only when
there is \emph{no} other equal definitions with body $a$ in the
environment. If there is, then the second rule apply.

In this case, we just ignore the first premise, as $a$ was already
type-checked. This rule is the key to incrementality.

Last rule, the application, is not very different from the usual one,
except that since there can be only a variable in the argument
position, we inline the Init rule and get rid of the second premise.

\slide

Let's delve a moment into a particular implementation detail: how to
choose names. We saw a quite unusual query to the environment before:
look for all variables with body exactly $a$. How to decide this query
efficiently?

We inspire once more from \textsf{git}, and give \emph{internal}
names, or keys, to all objects, that reflect their contents. We
introduce a morally injective hash function, from lists of keys to
keys, an efficient comparison function between two hashes, and a fresh
key generator.

Here are the introduction rules of the system: When we introduce a
definition $a$, we choose the hash of $a$ as a name for it, and do the
substitution in the rest of the term. When we introduce a declaration
(a product), we choose a new key for its name, and do the
substitution.

Therefore, the environment is then a function from a key to the
content of its definition (if it's a definition) and its type.

But let me finish with a remark: all of this is very static, content
is determined exactly by its syntactic shape. There seems to lack a
notion of computation into the calculus.

\slide

What would happen if we reintroduce computation, and in particular the
abstraction? We could not only declare logics but also develop its
metatheory, as we do in Pure Type Systems. For example, the weakening
property would have an existence, and we could even compute it since
it's constructively defined. In a sense, there is hope to relax the
very strict notion of equality implicit in the calculus, from
syntactic equality to ``semantic'' equality.

Moreover, we would be able to express functions that transform the
syntax, that is \emph{patches}. This opens a set of research
directions : are they valid with respect to a logic? what can we say
about dependencies between patches? Their commutation etc.

\end{document}
